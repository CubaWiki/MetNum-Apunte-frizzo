% -*- root: apunte-metodos.tex -*-

\section{Interpolación polinómica}
\label{section:interpolacion}

El problema de \textbf{interpolación} consiste en, dado un conjunto de puntos
$(x_0,f(x_0)), \dots, (x_n,f(x_n))$, encontrar una función $I$ tal que para
todo $i \in \{0,\dots,n\}$ se cumpla que $I(x_i) = f(x_i)$. Este problema
tiene múltiples aplicaciones; resulta útil, por ejemplo:
\begin{itemize}
\item cuando se tiene una serie de muestras discretas de alguna función y se
busca reconstruir la función a partir de las mismas, por ejemplo, para
aumentar la granularidad de las muestras agregando valores intermedios (esto
es útil en el caso de funciones que se encuentran tabuladas), o para realizar
una representación gráfica.
\item cuando se quiere estimar el valor de una función en un punto desconocido
a partir de los valores que toma en otros puntos.
\item para evaluar, derivar, integar, etc. una versión más simple de una
función complicada; por ejemplo, obteniendo un polinomio que tenga un
comportamiento similar.
\item en aplicaciones relacionadas con procesamiento multimedia, como
redimensionamiento de imágenes, \emph{slow motion} en videos, etc.
\end{itemize}

La forma más común de interpolación es la \textbf{interpolación polinómica},
donde la función interpolante es un polinomio, o una combinación de
polinomios. Esto se debe principalmente a que los polinomios son una clase de
funciones muy estudiada, y tienen múltiples propiedades deseables; por
ejemplo, son sencillos de evaluar, derivar e integrar.

\subsection{Interpolación con polinomio único}

\subsubsection{Polinomio interpolador de Lagrange}

Dado un conjunto de puntos $(x_0,f(x_0)), \dots, (x_n,f(x_n))$, el
\textbf{polinomio interpolador de Lagrange} en $x_0, \dots, x_n$ se define
como
\[ P(X) := \sum_{k=0}^n f(x_k) \cdot L_{k}(X)
    \qquad \text{donde} \qquad
    L_{k}(X) := \prod_{\substack{j=0 \\ j \neq k}}^n
    \frac{(X-x_j)}{(x_k-x_j)}. \]

Podemos observar que el polinomio de Lagrange interpola correctamente los
puntos, dado que para todos $k,i \in \{0,\dots,n\}$,
\[ L_k(x_i) = \begin{cases}
    1 & \text{si } k = i \\
    0 & \text{si } k \neq i
    \end{cases} \]
de donde, para todo $i \in \{0,\dots,n\}$, $\displaystyle P(x_i) =
    \sum_{\substack{k=0 \\ k \neq i}}^n f(x_k) \cdot L_{k}(x_i) +
    f(x_i) \cdot L_i(x_i) = f(x_i)$.

Una propiedad interesante del polinomio interpolador de Lagrange es que, si
$x_0, \dots, x_n \in [a,b]$ y $f \in \mathcal{C}^{n+1}([a,b])$, entonces
podemos dar, para cualquier punto intermedio entre las muestras utilizadas,
una aproximación del \textbf{error} cometido en la interpolación (es decir, de
la diferencia entre el resultado de evaluar en dicho punto el polinomio
obtenido y el valor correspondiente a la función $f$ que produjo los valores
a interpolar) en términos de la $n+1$-ésima derivada de $f$. En particular,
para todo $x \in [a,b]$ existe $\xi_x \in [a,b]$ tal que  
\[ f(x)
    = P(x) + \frac{f^{(n+1)}(\xi_x)}{(n+1)!} \cdot \prod_{i=0}^{n}(x-x_i) \]

Otra propiedad destacada del polinomio de Lagrange es su \textbf{unicidad}: es
el único polinomio de grado menor o igual que $n$ que interpola un conjunto de
puntos $(x_0, f(x_0)), \dots, (x_n, f(x_n))$. En efecto, sea $P$ el polinomio
interpolador de Lagrange en los puntos $x_0, \dots, x_n$, y supongamos que
existe otro polinomio $Q$, de grado menor o igual que $n$, tal que, para todo
$i \in \{0,\dots,n\}$, $Q(x_i) = f(x_i)$. Ahora bien, la función $Q(x) \in
\mathcal{C}^{n+1}([x_0,x_n])$, con $Q^{(n+1)}(x) \equiv 0$, y su polinomio
interpolador de Lagrange en los los puntos $x_0, \dots, x_n$ es precisamente
$P$. Así, podemos aplicar la fórmula del error, de donde, para todo
$x \in [x_0,x_n]$,
\[ Q(x) = P(x) + \frac{Q^{(n+1)}(\xi_x)}{(n+1)!} \cdot \prod_{i=0}^{n}(x-x_i)
    = P(x),\quad \text{para algún } \xi_x \in [x_0,x_n]. \]

\subsubsection{Diferencias divididas}
La forma de \textbf{diferencias divididas} es una manera alternativa de
escribir el polinomio interpolador de Lagrange. Presenta una ventaja importante
sobre la forma expuesta arriba; si se tiene el polinomio interpolador en
una serie de puntos $x_0, \dots, x_n$ y se agrega un nuevo punto $x_{n+1}$,
puede obtenerse un nuevo polinomio interpolador a partir del anterior, sin
necesidad de recomputarlo por completo.

Partimos de la siguiente definición recursiva:
\begin{itemize}
\item La \textbf{diferencia dividida de orden 0} en $x_j$ es, para $j = 0,
    \dots, n$:
    \[f[x_j] := f(x_j).\]
\item La \textbf{diferencia dividida de orden 1} en $x_j$ es, para $j = 0,
    \dots, n-1$:
    \[f[x_j,x_{j+1}]
        := \frac{f(x_{j+1}) - f(x_j)}{x_{j+1} - x_j}
        = \frac{f[x_{j+1}] - f[x_j]}{x_{j+1} - x_j}.\]
\item La \textbf{diferencia dividida de orden 2} en $x_j$ es, para $j = 0,
    \dots, n-2$:
    \[f[x_j,x_{j+1},x_{j+2}]
        := \frac{f[x_{j+1},x_{j+2}] - f[x_j,x_{j+1}]}{x_{j+2} - x_j}.\]
\item En general, la \textbf{diferencia dividida de orden $\bm{k}$} en $x_j$
    es, para $j = 0, \dots, n-k$:
    \[f[x_j,\dots,x_{j+k}]
        := \frac{f[x_{j+1},\dots,x_{j+k}] - f[x_j,\dots,x_{j+k-1}]}{x_{j+k} - x_j}.\]
\end{itemize}

Afirmamos que, si $P$ es el polinomio interpolador para los puntos $x_0,
\dots, x_n$, entonces
\[ P(X) = \sum_{i=0}^n f[x_0,\dots,x_i]
    \left( \prod_{j=0}^{i-1} (X - x_j) \right) \]
es decir,
\[ \begin{aligned}
P(X) = f[x_0] &+ f[x_0,x_1] (X - x_0) + f[x_0,x_1,x_2] (X - x_0) (X - x_1) \\
       &+ \dots + f[x_0,x_1,\dots,x_n] (X - x_0) (X - x_1) \dots (X - x_n).
\end{aligned} \]

Este resultado, que se conoce a veces como \textbf{forma de Newton} del
polinomio interpolador, puede demostrarse por inducción en la cantidad de
puntos interpolados. El polinomio queda escrito en una serie de términos que
dependen de una cantidad incremental de los puntos interpolados; más
específicamente, el término $i$-ésimo del polinomio depende solo de los
valores $x_0,\dots,x_i$. Así, para agregar un nuevo punto a interpolar, basta
con agregar un nuevo término al polinomio. Este término, además, se puede
calcular de forma eficiente, ya que su coeficiente es la
diferencia dividida $f[x_0, \dots, x_{n+1}]$, que, gracias a la estructura
recursiva de las diferencias divididas, se puede computar reutilizando
resultados anteriores.

\begin{figure}[H]
\begin{center}
\begin{tikzpicture}
\node (A1)               {$f[x_0]$};
\node (A2) [below=of A1] {$f[x_1]$};
\node (A3) [below=of A2] {$f[x_2]$};
\node (A4) [below=of A3, draw] {$f[x_3]$};

\node (B1) [right=1cm of $(A1.east)!0.5!(A2.east)$] {$f[x_0,x_1]$};
\node (B2) [right=1cm of $(A2.east)!0.5!(A3.east)$] {$f[x_1,x_2]$};
\node (B3) [right=1cm of $(A3.east)!0.5!(A4.east)$, draw] {$f[x_2,x_3]$};

\node (C1) [right=1cm of $(B1.east)!0.5!(B2.east)$] {$f[x_0,x_1,x_2]$};
\node (C2) [right=1cm of $(B2.east)!0.5!(B3.east)$, draw] {$f[x_1,x_2,x_3]$};

\node (D1) [right=1cm of $(C1.east)!0.5!(C2.east)$, draw] {$f[x_0,x_1,x_2,x_3]$};

\draw (A1) -> (B1.west);
\draw (A2) -> (B1.west);
\draw (A2) -> (B2.west);
\draw (A3) -> (B2.west);
\draw (A3) -> (B3.west);
\draw (A4) -> (B3.west);
\draw (B1) -> (C1.west);
\draw (B2) -> (C1.west);
\draw (B2) -> (C2.west);
\draw (B3) -> (C2.west);
\draw (C1) -> (D1.west);
\draw (C2) -> (D1.west);
\end{tikzpicture}
\end{center}

\label{fig:dif-divididas}
\caption{Diferencias divididas que es necesario computar para agregar un
cuarto punto a un conjunto de tres puntos ya interpolados.}
\end{figure}

\subsubsection{Método de Neville}

El polinomio interpolador para los puntos $x_1, \dots, x_n$ admite
también una formulación recursiva, en términos de dos polinomios que
interpolan, cada uno, $n-1$ puntos. Más específicamente, para cualesquiera
$0 \leq i,j \leq n$, con $i \neq j$, se tiene que
\[ P(X) = \frac{(X - x_j) \cdot P_j(X)
    - (X - x_i) \cdot P_i(X)}{x_i - x_j}, \]
donde $P_i$ denota al polinomio interpolador para los puntos
$x_0,\dots,x_{i-1},x_{i+1},\dots,x_n$, y $P_j$ al polinomio interpolador para
$x_0,\dots,x_{j-1},x_{j+1},\dots,x_n$.

No es difícil ver que este polinomio interpola correctamente todos los puntos:
\begin{itemize}
\item En $x_i$, tenemos que $P(x_i)
    = \frac{(x_i - x_j) \cdot P_j(x_i) - (x_i - x_i) \cdot P_i(x_i)}{x_i - x_j}
    = P_j(x_i) = f(x_i)$.
\item Análogamente, en $x_j$, tenemos que $P(x_j)
    = \frac{(x_j - x_j) \cdot P_j(x_j) - (x_j - x_i) \cdot P_i(x_j)}{x_i - x_j}
    = P_i(x_j) = f(x_j)$.
\item En $x_k$, con $k \neq i, j$, tenemos que $P(x_k)
    = \frac{(x_k - x_j) \cdot P_j(x_k) - (x_k - x_i) \cdot P_i(x_k)}{x_i - x_j}
    = f(x_k) \cdot \frac{(x_k - x_j) - (x_k - x_i)}{x_i - x_j} = f(x_k)$.
\end{itemize}

Esta escritura recursiva da origen al \textbf{método de Neville}, que es otra
manera de construir polinomios interpoladores de forma incremental, permitiendo
obtener un polinomio interpolador para $n+1$ puntos a partir
de uno que interpola un subconjunto de $n$ de estos puntos.

\begin{figure}[H]
\begin{center}
\begin{tikzpicture}
\node (A1)               {$P_{0,0}(X) \equiv f(x_0)$};
\node (A2) [below=of A1] {$P_{1,1}(X) \equiv f(x_1)$};
\node (A3) [below=of A2] {$P_{2,2}(X) \equiv f(x_2)$};
\node (A4) [below=of A3, draw] {$P_{3,3}(X) \equiv f(x_3)$};
% \node (A5) [below=of A4, draw] {$P_{4,4}(X) \equiv f(x_4)$};

\node (B1) [right=1cm of $(A1.east)!0.5!(A2.east)$] {$P_{0,1}(X)$};
\node (B2) [right=1cm of $(A2.east)!0.5!(A3.east)$] {$P_{1,2}(X)$};
\node (B3) [right=1cm of $(A3.east)!0.5!(A4.east)$, draw] {$P_{2,3}(X)$};
% \node (B4) [right=1cm of $(A4.east)!0.5!(A5.east)$, draw] {$P_{3,4}(X)$};

\node (C1) [right=1cm of $(B1.east)!0.5!(B2.east)$] {$P_{0,2}(X)$};
\node (C2) [right=1cm of $(B2.east)!0.5!(B3.east)$, draw] {$P_{1,3}(X)$};
% \node (C3) [right=1cm of $(B3.east)!0.5!(B4.east)$, draw] {$P_{2,4}(X)$};

\node (D1) [right=1cm of $(C1.east)!0.5!(C2.east)$, draw] {$P_{0,3}(X)$};
% \node (D2) [right=1cm of $(C2.east)!0.5!(C3.east)$, draw] {$P_{1,4}(X)$};

% \node (E1) [right=1cm of $(D1.east)!0.5!(D2.east)$, draw] {$P_{0,4}(X)$};

\draw (A1) -> (B1.west);
\draw (A2) -> (B1.west);
\draw (A2) -> (B2.west);
\draw (A3) -> (B2.west);
\draw (A3) -> (B3.west);
\draw (A4) -> (B3.west);
% \draw (A4) -> (B4.west);
% \draw (A5) -> (B4.west);
\draw (B1) -> (C1.west);
\draw (B2) -> (C1.west);
\draw (B2) -> (C2.west);
\draw (B3) -> (C2.west);
% \draw (B3) -> (C3.west);
% \draw (B4) -> (C3.west);
\draw (C1) -> (D1.west);
\draw (C2) -> (D1.west);
% \draw (C2) -> (D2.west);
% \draw (C3) -> (D2.west);
% \draw (D1) -> (E1.west);
% \draw (D2) -> (E1.west);
\end{tikzpicture}
\end{center}

\label{fig:met-neville}
\caption{Extensión de un polinomio que interpola los puntos $x_0, \dots, x_3$
para interpolar también el punto $x_4$. La notación $P_{i,j}$ indica, para $0
\leq i \leq j \leq 4$, el polinomio interpolador en los puntos
$x_i,x_{i+1},\dots,x_j$.}
\end{figure}

\subsection{Interpolación fragmentaria}
Cuando se busca interpolar una cantidad de muestras de cierto tamaño, el
polinomio interpolador empieza a dar cuenta de algunos problemas. Esto se debe
a que en general, a medida que crece la cantidad de valores a interpolar,
aumenta también el grado del polinomio interpolador. El inconveniente es que
los polinomios de grado elevado tienden a oscilar demasiado, lo cual suele no
ajustarse bien al comportamiento de la función que produjo las muestras y, por
lo tanto es un comportamiento poco deseado al interpolar.

Una alternativa ante esta situación es la \textbf{interpolación fragmentaria},
que consiste en utilizar polinomios distintos, de menor grado, en diferentes
fragmentos del intervalo donde se quiere interpolar. El resultado ya no será
un polinomio, sino una función compuesta de muchos polinomios ``pegados'' en
los extremos.

En general, buscaremos definir $n$ polinomios distintos, $S_1, \dots, S_n$,
uno para cada par de puntos consecutivos entre los valores a interpolar. La
idea es que estos polinomios cumplan $S_i(x_{i-1}) = f(x_{i-1})$ y $S_i(x_i)
= f(x_i)$, para $i = 1, \dots, n$. Luego, podremos construir la función
\[ S(x) := \begin{cases}
    S_1(x) & \text{si } x \in [x_0,x_1) \\
    S_2(x) & \text{si } x \in [x_1,x_2) \\
    \quad \vdots \\
    S_n(x) & \text{si } x \in [x_{n-1}, x_n].
\end{cases} \]

\subsubsection{Interpolación fragmentaria lineal}

La forma más sencilla de interpolación fragmentaria es la
\textbf{interpolación lineal}, donde cada uno de los $S_i$ es un polinomio de
grado menor o igual que 1, que interpola correctamente en los puntos $x_{i-1}$
y $x_i$\footnote{Cada $S_i$ es, así, el polinomio interpolador de Lagrange
para los puntos $x_{i-1}, x_i$.}. Los $S_i$ se definen, para $i = 1, \dots,
n$, como
\[ S_i(X) := f(x_{i-1})
    + \frac{f(x_i) - f(x_{i-1})}{x_i-x_{x-1}} \cdot (X - x_{i-1}). \]

% \begin{center}
% \begin{tikzpicture}
% \begin{axis}[
%   axis lines=middle,
%   clip=false,
%   ymin=0,
%   xticklabels={$x_0$,$x_1$,$x_2$},
%   yticklabels=\empty,
% ]
% % \addplot+[mark=none,samples=200,unbounded coords=jump] {sqrt(x)};
% \addplot [patch, patch type=quadratic spline] coordinates {
%     (0, 4)
%     (1, 3)
%     (1, 3)
%     (2, 4)
%     (3, 6)
%     (4, 7)
%     (5, 5)
%     (6, 4)
%     (7, 2)
%     (8, 3)
%     (9, 3)
%     (10, 4)
% };
% \end{axis}
% \end{tikzpicture}
% \end{center}

La interpolación lineal es sencilla, fácil de calcular (incluso en forma
manual) y en muchas ocasiones resulta suficiente para el problema que se
busca resolver. Sin embargo, la función que se obtiene no es derivable en
los puntos $x_i$, donde presenta un aspecto ``puntiagudo'' que no resulta
natural. Este problema puede resolverse utilizando polinomios de mayor grado.

\subsubsection{Interpolación fragmentaria cuadrática}
Si cada $S_i$ se define para ser un polinomio \textbf{cuadrático}, se tienen más
parámetros para definir, que se pueden aprovechar para que la función $S$
obtenida sea derivable en todo su dominio. La idea es definir los $S_i$ en
la forma
\[ S_i(X) := a_i (X - x_i)^2 + b_i (X - x_i) + c_i \]
y determinar valores para los $a_i$, $b_i$ y $c_i$ de modo que se cumplan las
siguientes condiciones:

\begin{enumerate}[label=(\roman*)]
\item $S$ es interpoladora:
    \begin{itemize}
    \item $S_i(x_{i-1}) = f(x_{i-1})$, para $i = 1, \dots, n$ ($n$ ecuaciones).
    \item $S_n(x_n) = f(x_n)$ ($1$ ecuación).
    \end{itemize}
\item $S$ es continua:
    $S_i(x_i) = f(x_i)$, para $i = 1, \dots, n-1$ ($n-1$ ecuaciones).
\item $S$ es derivable:
    $S_i'(x_i) = S_{i+1}'(x_i)$, para $i = 1, \dots, n-1$ ($n-1$ ecuaciones).
\end{enumerate}

Se tiene un sistema de $3n$ incógnitas y $3n-1$ ecuaciones, lo cual deja una
única ecuación libre para pedir alguna propiedad adicional. En general, esta
se utiliza para controlar el comportamiento de la derivada en alguno de los
extremos $x_0$ o $x_n$, con el inconveniente de que se obtiene una solución
asimétrica, ya que es imposible pedir condiciones simultáneamente sobre los
dos extremos y mantener la certeza de que el sistema resulta compatible.

\subsubsection{\emph{Splines} cúbicos}
La utilización de polinomios \textbf{cúbicos} para definir cada $S_i$ resuelve
el problema recién mencionado, y permite a su vez obtener una función
interpoladora que es dos veces derivable en todo su dominio. Para lograrlo,
los $S_i$ se definen en la forma
\[ S_i(X) := a_i (X - x_i)^3 + b_i (X - x_i)^2 + c_i (X - x_i) + d_i \]
y los valores para cada $a_i$, $b_i$, $c_i$ y $d_i$ se determinan de modo tal
que:

\begin{enumerate}[label=(\roman*)]
\item $S$ es interpoladora:
    \begin{itemize}
    \item $S_i(x_{i-1}) = f(x_{i-1})$, para $i = 1, \dots, n$ ($n$ ecuaciones).
    \item $S_n(x_n) = f(x_n)$ ($1$ ecuación).
    \end{itemize}
\item $S$ es continua:
    $S_i(x_i) = f(x_i)$, para $i = 1, \dots, n-1$ ($n-1$ ecuaciones).
\item $S$ es derivable:
    $S_i'(x_i) = S_{i+1}'(x_i)$, para $i = 1, \dots, n-1$ ($n-1$ ecuaciones).
\item $S$ es dos veces derivable:
    $S_i''(x_i) = S_{i+1}''(x_i)$, para $i = 1, \dots, n-1$ ($n-1$ ecuaciones).
\end{enumerate}

El sistema que resulta tiene $4n$ incógnitas y $4n - 2$ ecuaciones, con lo que
pueden agregarse dos nuevas ecuaciones, para condicionar el comportamiento en
los puntos frontera, $x_0$ y $x_n$. Esto puede hacerse de dos formas
alternativas:

\begin{enumerate}[label=(\roman*), resume]
\item \begin{enumerate}[label=(\alph*)]
    \item Frontera sujeta: \begin{itemize}
        \item $S_1'(x_0) = f'(x_0)$.
        \item $S_n'(x_n) = f'(x_n)$.
        \end{itemize}
    \item Frontera natural: $S_1''(x_0) = S_n''(x_n) = 0$.
\end{enumerate}
\end{enumerate}

En ambos casos, el sistema de ecucaciones que se obtiene es tridiagonal y
estrictamente diagonal dominante. Esto permite resolverlo computacionalmente
de forma eficiente, y asegura que siempre tiene solución única.
