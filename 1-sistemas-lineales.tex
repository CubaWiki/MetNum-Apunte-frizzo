% -*- root: apunte-metodos.tex -*-

\section{Resolución directa de sistemas de ecuaciones lineales}
\label{section:sistemas-lineales}

Un \textbf{sistema de ecuaciones lineales} es un conjunto de ecuaciones de la
forma
\[ \left\lbrace \begin{matrix}
    a_{1,1} x_1 &+& a_{1,2} x_2 &+& \cdots &+& a_{1,n} x_n & = & b_1    \\
    a_{2,1} x_1 &+& a_{2,2} x_2 &+& \cdots &+& a_{2,n} x_n & = & b_2    \\
    \vdots     &+& \vdots     &+&     &+& \vdots     & = & \vdots \\
    a_{n,1} x_1 &+& a_{n,2} x_2 &+& \cdots &+& a_{n,n} x_n & = & b_n    \\
\end{matrix} \right. \]
donde los $a_{i,j}$ y los $b_i$ son números reales.

Los sistemas de ecuaciones lineales admiten también la representación
matricial
\[ \mat{A} \cdot x = b \]
donde $\mat{A} \in \reals ^{n \times n}$, $x, b \in \reals^{n}$. Esta
representación nos facilitará tanto su comprensión como su tratamiento
computacional.

Las variables $x_1, \dots, x_n$ se denominan las \textbf{incógnitas} del
sistema.
Una \textbf{solución} de un sistema de ecuaciones lineales es un conjunto de
valores para las incógnitas que satisfacen simultáneamente todas las
ecuaciones.

Un sistema de ecuaciones lineales puede no tener solución, tener solución
única, o tener infinitas soluciones. Si la matriz asociada al sistema es
inversible (o, lo que es lo mismo, sus columnas son linealmente
independientes)
la solución será única. Si, por el contrario, la matriz es singular,
podría pasar que el sistema no tenga solución o que tenga infinitas de ellas.

Un sistema de la forma $\mat{A} \cdot x = 0$ se denomina \textbf{homogéneo}.
Las soluciones de un sistema homogéneo forman un subespacio vectorial. Además,
el conjunto de soluciones de cualquier sistema $\mat{A} \cdot x = b$ puede
obtenerse sumando una solución particular del mismo a las soluciones del
sistema homogéneo asociado, $\mat{A} \cdot x = 0$.

En esta sección, expondremos dos métodos para obtener computacionalmente la
solucion de un sistema de ecuaciones lineales, en el caso de que esta exista y
sea única. La resolución de estos sistemas es un problema importante y
frecuente en el análisis numérico, ya que estos son útiles a la hora de
modelar matemáticamente el comportamiento de problemas provenientes de
diversas disciplinas, como la física y la ingeniería, para ser tratados en
forma computacional. En muchos de estos modelos aparecen ecuaciones
que, o bien son lineales, o pueden aproximarse bien mediante ecuaciones
lineales. Estos sistemas también aparecen en la resolución de ecuaciones
diferenciales, que son cruciales para muchas disciplinas.

\subsection{Resolución de sistemas sencillos}

Decimos que una matriz $\mat{A} \in \reals^{n \times n}$ es:
\begin{itemize}[itemsep=0em]
\item \textbf{diagonal}, si $\fall{1 \leq i,j \leq n} i \neq j \Rightarrow a_{i,j} = 0$;
    es decir, todos los elementos fuera de la diagonal son nulos.
\item \textbf{triangular inferior}, si $\fall{1 \leq i,j \leq n} i < j \Rightarrow a_{i,j} = 0$;
    es decir, todos los elementos por encima de la diagonal son nulos.
\item \textbf{triangular superior}, si $\fall{1 \leq i,j \leq n} i > j \Rightarrow a_{i,j} = 0$;
    es decir, todos los elementos por debajo de la diagonal son nulos.
\end{itemize}

Existen ciertos sistemas de ecuaciones que se pueden resolver algorítmicamente
de forma sencilla. Por ejemplo, si el sistema tiene asociada una matriz
diagonal, entonces sus ecuaciones son de la forma
\[ \left\lbrace \begin{aligned}
    a_{1,1} x_1 &= b_1 \\
    a_{2,2} x_2 &= b_2 \\
    \vdots \\
    a_{n,n} x_n &= b_n, \\
\end{aligned} \right. \]
de donde pueden despejarse fácilmente valores para cada $x_i$. Se puede notar
que existirá una solución única si y solo si $\fall{1 \leq i \leq n} a_{i,i}
\neq 0$, condición que equivale a que la matriz sea inversible. En tal caso,
el costo de hallarla será lineal en la cantidad de ecuaciones, es decir,
$\ord{n}$.

Por otra parte, si la matriz del sistema es triangular superior, puede
aplicarse un algoritmo llamado \textbf{sustitución hacia atrás}
(\emph{backward substitution}). El mismo consiste en comenzar a partir de la
última ecuación, que tiene la forma
\[ a_{n,n} x_n = b_n, \]
de donde es sencillo despejar un valor para $x_n$. Luego, en la ecuación
anterior, que tiene la forma
\[ a_{n-1,n-1} x_{n-1} + a_{n-1,n} x_n = b_{n-1}, \]
puede reemplazarse el valor hallado para $x_n$, obteniendo así un valor para
$x_{n-1}$. Este procedimiento se repite con las ecuaciones superiores,
hasta haber despejado todas las incógnitas del sistema. Al igual que en el
caso anterior, esto será posible si y solo si todos los elementos de la
diagonal son no nulos; en caso contrario, el sistema no tiene solución única.

Así, la solución hallada mediante el algoritmo de sustitución hacia atrás
puede describirse mediante las siguientes ecuaciones:
\[ \left\lbrace \begin{aligned}
    x_{n} &= \frac{b_{n}}{a_{n,n}} \\
    x_{n-1} &= \frac{b_{n-1} - a_{n-1,n}\cdot x_{n}}{a_{n-1,n-1}} \\
    \vdots \\
    x_{1} &= \frac{b_{1} - a_{1,2}\cdot x_{2} - \dots - a_{1,n}\cdot x_{n}}{a_{1,1}} \\
\end{aligned} \right. \]

Si la matriz del sistema es triangular inferior, se puede modificar el
algoritmo para comenzar despejando en la primera ecuación. A este
procedimiento se lo conoce como \textbf{sustitución hacia adelante}
(\emph{forward substitution}).

Tanto el algoritmo de sustitución hacia adelante como el algoritmo de
sustitución hacia atrás tienen un costo cuadrático en la cantidad de
ecuaciones del sistema ($\ord{n^2}$).

\subsection{Eliminación gaussiana}

Decimos que dos sistemas de ecuaciones son \textbf{equivalentes} si tienen el
mismo conjunto de soluciones. El algoritmo de \textbf{eliminación gaussiana}
consiste en transformar un sistema de ecuaciones cualquiera en otro
equivalente, pero que se encuentre en forma triangular superior. De esta
forma,
puede aplicarse el procedimiento de sustitución hacia atrás para encontrar
las soluciones del sistema original.

Para transformar un sistema en otro equivalente, se aplica una serie de
operaciones sobre las ecuaciones del mismo, que no modifican su conjunto
de soluciones. Estas operaciones son las siguientes:
\begin{enumerate}[label=(\arabic*),itemsep=0em]
\item Intercambiar el orden de dos ecuaciones.
\item Multiplicar una ecuación por una constante $\lambda \in \reals$ no nula.
\item Sumar a una ecuación el resultado de multiplicar otra por una constante
$\lambda \in \reals$.
\end{enumerate}

Con el sistema en forma matricial, dichas operaciones entre ecuaciones se
traducen a operaciones entre filas de la matriz, y pueden representarse como
el producto a izquierda por ciertas matrices particulares, llamadas
\textbf{matrices elementales}. Una matriz elemental es cualquiera de las
siguientes:
\begin{enumerate}[label=(\arabic*)]
\item Una matriz $\mat{P}$, obtenida de permutar filas o columnas de la
    identidad; a esta matriz se la llama \textbf{matriz de permutación}.
    Si $\mat{A}$ es una matriz cualquiera, entonces $\mat{P}\cdot\mat{A}$ es
    una permutación de las filas de $\mat{A}$, y $\mat{A}\cdot\mat{P}$ es una
    permutación de las columnas de $\mat{A}$.
\item Una matriz $\mat{E_1} = \begin{bmatrix}
    1      & \cdots & 0       & \cdots & 0      \\
    \vdots & \ddots & \vdots  &        & \vdots \\
    0      & \cdots & \lambda & \cdots & 0      \\
    \vdots &        & \vdots  & \ddots & \vdots \\
    0      & \cdots & 0       & \cdots & 1      \\
\end{bmatrix}$, obtenida de cambiar el 1 de la $i$-ésima fila de la
identidad por un valor $\lambda$ cualquiera. Si $\mat{A}$ es una matriz
cualquiera, multiplicar a izquierda (a derecha) por $\mat{E_1}$ multiplica por
$\lambda$ la $i$-ésima fila (columna) de $\mat{A}$.
\item Una matriz $\mat{E_2} = \begin{bmatrix}
    1      & \cdots  & 0      & \cdots & 0      \\
    \vdots & \ddots  & \vdots &        & \vdots \\
    0      & \cdots  & 1      & \cdots & 0      \\
    \vdots & \lambda & \vdots & \ddots & \vdots \\
    0      & \cdots  & 0      & \cdots & 1      \\
\end{bmatrix}$, obtenida de cambiar un 0 de la identidad, ubicado
en la posición $i,j$, por un valor $\lambda$ cualquiera. Si $\mat{A}$ es una
matriz cualquiera, multiplicar a izquierda (a derecha) por $\mat{E_2}$
la suma $\lambda$ veces la fila $j$-ésima (la columna $i$-ésima) a la
fila $i$-ésima (la columna $j$-ésima) de $\mat{A}$.
\end{enumerate}

Las matrices elementales son inversibles, y el producto por la inversa de una
matriz elemental revierte la operación que esta realiza sobre las filas de una
matriz cualquiera.

El algoritmo de eliminación gaussiana opera sobre la \textbf{matriz extendida}
del sistema, que es la matriz
\[ \mat{\tilde A} = \mleft[ \begin{array}{cccc|c}
    a_{1,1} & a_{1,2} & \dots  & a_{1,n} & b_{1} \\
    a_{2,1} & a_{2,2} & \dots  & a_{2,n} & b_{2} \\
    \vdots  & \vdots  & \ddots & \vdots  & \vdots \\
    a_{n,1} & a_{n,2} & \dots  & a_{n,n} & b_{n} \\
\end{array} \mright]. \]
Como cada iteración efectúa cambios sobre la matriz $\mat{\tilde A}$,
utilizaremos la notación $\mat{\tilde A}^{(k)}$ para referirnos al resultado
luego de la $k$-ésima iteración del proceso, mientras que con $a^{(k)}_{i,j}$
y $b^{(k)}_i$ haremos referencia a cada uno de sus elementos.

La idea del algoritmo es aplicar operaciones de filas en forma consecutiva
hasta llevar $\mat{\tilde A}$ a una forma triangular superior. El método itera
sobre las columnas de la matriz, buscando en cada paso colocar ceros en los
lugares que se encuentran debajo de la diagonal. Es decir, el invariante del
algoritmo es que, en la $k$-ésima iteración, todas las columnas hasta la $k -
1$ tienen ceros debajo de la diagonal, asegurando que, tras $k$ iteraciones,
la matriz quedará en forma triangular superior.

Más precisamente, en la $k$-ésima iteración, se resta a todas las filas a
partir de la $k + 1$ un múltiplo de la fila $k$-ésima, con un factor
$m^{(k)}_i$ elegido convenientemente para cada fila $i$. Esto significa que,
para todo $i = k+1,\dots,n$, los coeficientes de la fila $i$-ésima quedarán
\[ a^{(k)}_{i,j} = a^{(k-1)}_{i,j} - m^{(k)}_i \cdot a^{(k-1)}_{k,j}, \]
y como se quiere dejar un $0$ en la columna $k$-ésima, es decir,
$a^{(k)}_{i,k} = 0$, debe tomarse, para cada fila $i$, el multiplicador
\[ m_{i,k} = \frac{a^{(k-1)}_{i,k}}{a^{(k-1)}_{k,k}}. \]

Es importante notar que solo es posible efectuar el $k$-ésimo paso del
algoritmo si $a^{(k-1)}_{k,k} = a_{k,k} \neq 0$, es decir, si el $k$-ésimo
elemento de la diagonal no es nulo. Se puede relajar ligeramente esta
hipótesis, admitiendo $a^{(k-1)}_{k,k} = 0$, si todos los demás elementos
de esa columna debajo de la diagonal también son nulos; en ese caso
directamente se puede omitir la $k$-ésima iteración. En cualquier otro caso,
el algoritmo falla.

Como cada paso del algoritmo coloca ceros debajo de la diagonal en la columna
$k$-ésima, y no modifica los ceros que fueron ubicados en otras columnas por
los pasos previos, la matriz $\mat{\tilde A}^{(n-1)}$ que se obtiene tras
$n-1$ iteraciones del proceso es triangular superior.

A continuación se presenta el algoritmo en forma de pseudocódigo. Analizando
el mismo, se puede ver que su complejidad es de orden cúbico ($\ord{n^3}$).

\begin{algorithm}[H]
\caption{Algoritmo de eliminación gaussiana}
\label{algo:eliminacion-gauss}

\Input{$\mat{A} \in \reals^{n \times n}$ (con filas $F_1,\dots,F_n$) y $b \in \reals^n$.}
\Output{$x \in \reals^n$ tal que $\mat{A} \cdot x = b$.}
\For {$k = 1,\dots,n-1$} {
    \eIf {$a_{k,k} \neq 0$} {
        \For {$i = k+1,\dots,n$} {
            $m_{i,k} \gets \dfrac{a_{i,k}}{a_{k,k}}$ \;
            $F_i \gets F_i - m_{i,k} \cdot F_k$ \;
        }
    }
    {
        \If {$a_{i,k} \neq 0$ para algún $i \in {k+1,...,n}$} {
            \Error
        }
    }
}
\end{algorithm}

\subsubsection{Pivoteo}

En cada paso del algoritmo de eliminación gaussiana, llamamos \textbf{pivote}
al elemento de la diagonal sobre el cual estamos trabajando (en el paso
$k$-ésimo, el pivote es $a^{(k-1)}_{k,k}$). La técnica de \textbf{pivoteo}
consiste en realizar operaciones sobre la matriz, intercambiando sus filas
(o sus columnas) para modificar el pivote sin alterar las soluciones
del sistema asociado. Esto puede ser deseable por dos razones:

\begin{enumerate}
\item Como se vio anteriormente, si en algún paso del algoritmo el pivote
    es cero pero hay un elemento no nulo debajo de él, el procedimiento
    falla. Gracias al pivoteo, puede cambiarse el pivote por un elemento no
    nulo y proseguir con el algoritmo. Se puede demostrar que
    toda matriz admite eliminación gaussiana con pivoteo, incluso aquellas
    para las cuales falla la versión sin pivoteo.
\item Debido a que la computadora trabaja con arimética finita (una
    aproximación discreta de los números reales), durante las operaciones se
    suele perder precisión en los resultados. Es deseable que los algoritmos
    eviten que estos errores se amplifiquen durante iteraciones posteriores,
    propiedad que se conoce como \emph{estabilidad numérica}. En el caso de
    la eliminación gaussiana, la estabilidad numérica mejora cuando el
    pivote tiene un valor absoluto grande, por lo que se puede utilizar
    pivoteo para intercambiarlo por uno de mayor valor absoluto y así mejorar
    la precisión de los resultados obtenidos.
\end{enumerate}

Si se quiere aplicar eliminación gaussiana con pivoteo, es necesario
determinar un criterio para elegir el pivote en cada iteración. Existen
principalmente dos formas de hacerlo.
\begin{itemize}
\item El \textbf{pivoteo parcial} consiste en intercambiar el pivote por un
    elemento de la misma columna, considerando el propio pivote y los
    elementos que se encuentran por debajo de él, y eligiendo el de mayor
    valor absoluto. Por lo tanto, se lleva a cabo intercambiando dos filas de
    la matriz. Esta técnica solo requiere considerar, a lo sumo, $n$
    posibles valores para el pivote. Garantiza que se elegirá un pivote no
    nulo (a menos que el elemento de la diagonal y todos los que estén debajo
    sean nulos), y permite mejorar la estabilidad numérica.
\item El \textbf{pivoteo completo} considera toda la submatriz que falta
    reducir, eligiendo como pivote al elemento de mayor valor absoluto. Se
    lleva a cabo intercambiando dos filas y dos columnas de la matriz
    (intercambiar columnas equivale a alterar el orden de las variables del
    sistema, por lo que los intercambios de columnas deberán ser revertidos
    en la solución que se obtenga).
    Esta técnica permite mejorar aún más la estabilidad numérica, pero
    es poco utilizada por resultar considerablemente menos eficiente, ya que
    la búsqueda del pivote tiene una complejidad cuadrática.
\end{itemize}

\subsection{Factorización LU}

Dada una matriz $\mat{A} \in \reals^{n \times n}$, una factorización LU para
$\mat{A}$ es una escritura:
\[ \mat{A} = \mat{L} \cdot \mat{U} \]
donde $\mat{L} \in \reals^{n \times n}$ es triangular inferior con unos en
la diagonal y $\mat{U} \in \reals^{n \times n}$ es triangular superior.

Dada la factorización LU de una matriz $\mat{A}$, resolver un sistema de
ecuaciones $\mat{A} \cdot x = b$ asociado resulta sencillo. En primer lugar,
se puede notar que $\mat{L}$ es inversible, ya que es triangular inferior
sin ceros en la diagonal. Entonces,
\[ \begin{aligned}
    \mat{A} \cdot x &= b &\text{sii} \\
    \mat{L} \cdot \mat{U} \cdot x &= b &\text{sii} \\
    \mat{U} \cdot x &= \mat{L}^{-1} \cdot b &\text{sii} \\
    \mat{U} \cdot x &= y \\
\end{aligned} \]
donde $y = \mat{L}^{-1} \cdot b$ o, lo que es lo mismo, $y$ es la única
solución del sistema de ecuaciones $\mat{L} \cdot y = b$. Por lo tanto,
basta con resolver consecutivamente dos sistemas de ecuaciones:
\begin{itemize}
\item $\mat{L} \cdot y = b$,
\item $\mat{U} \cdot x = y$.
\end{itemize}
Como ambas matrices son triangulares, los sistemas se pueden resolver con
un costo $\ord{n^2}$, usando los algoritmos de sustitución hacia adelante y
hacia atrás, respectivamente.

Hallar la factorización LU, por su parte, tiene una complejidad de
$\ord{n^3}$ (como se verá luego con más detalle).
Por lo tanto, si se la quiere utilizar para resolver un sistema
desde cero, el costo es el mismo que el de aplicar eliminación gaussiana.
La ventaja de la factorización LU aparece si se quieren resolver varios
sistemas de ecuaciones que comparten la matriz asociada; es decir, si se
tiene una matriz $\mat{A}$ y una familia de vectores $b_1, \dots, b_k$, y se
quieren hallar soluciones para todos los sistemas $\mat{A} \cdot x = b_i$ con
$i \in \{1, \dots, k\}$.
En este caso, usando eliminación gaussiana, habría que pagar un costo
$\ord{n^3}$ para resolver cada sistema. Es mucho más eficiente calcular
primero la factorización LU de $\mat{A}$, pagando el costo $\ord{n^3}$
solo una vez, y luego resolver cada sistema con un costo $\ord{n^2}$.

La factorización LU está intrínsecamente relacionada al algoritmo de
eliminación gaussiana.
Esto se debe a que la matriz $\mat{U}$ puede interpretarse como el resultado
de aplicar eliminación gaussiana (sin pivoteo) sobre $\mat{A}$, mientras
que $\mat{L}$ es el producto de las matrices elementales que representan a
las operaciones involucradas en el proceso.

Para demostrar este resultado, recordemos lo que sucede en el paso $k$-ésimo
del algoritmo de eliminación gaussiana: para cada $i \in \{k+1,\dots,n\}$,
se calcula un multiplicador $m_{i,k}$ y se realiza la operación de filas
$F_i \gets F_i - m_{i,k} \cdot F_k$.\footnote{En el caso en que para cierta
columna $k$, el elemento de la diagonal y todos los que están debajo de él
sean nulos, el algoritmo no hace nada; en ese caso puede considerarse
que todos los $m_{i,k}$ valen $0$.}
Esta operación se puede representar como
el producto a izquierda por una matriz elemental $\mat{M}_{i,k}$, que es
similar a la identidad, pero tiene el valor $-m_{i,k}$ en la posición $i,k$.
Por lo tanto, considerando todas las operaciones de filas, podemos sintetizar
el $k$-ésimo paso de la siguiente manera (por simplicidad trabajaremos con la
matriz $\mat{A}$, en lugar de usar la matriz extendida $\mat{\tilde A}$):
\[ \begin{aligned}
    \mat{A}^{(k)}
    &= \mat{M}_{n,k} \cdot \, \dots \, \cdot \mat{M}_{k+1,k} \cdot \mat{A}^{(k-1)} \\
    &= \mat{M}_{k} \cdot \mat{A}^{(k-1)}.
\end{aligned} \]
La matriz $\mat{M}_{k}$, que es el producto de las matrices elementales
de cada operación de filas, es similar a la identidad pero en la $k$-ésima
columna tiene, debajo de la diagonal, los valores
$-m_{k+1,k}, \dots, -m_{n,k}$. Además, $\mat{M}_{k}$ es inversible; su inversa
es muy parecida, pero los $m_{i,k}$ no están cambiados de signo.

De forma similar, podemos sintetizar todo el algoritmo de eliminación
gaussiana como
\[ \begin{aligned}
    \mat{A}^{(n-1)}
    &= \mat{M}_{n-1} \cdot \, \dots \, \cdot \mat{M}_{1} \cdot \mat{A}^{(0)} \\
    &= \mat{M} \cdot \mat{A}^{(0)} \\
    &= \mat{M} \cdot \mat{A}.
\end{aligned} \]

Llamaremos $\mat{U}$ a la matriz $\mat{A}^{(n-1)}$, que es triangular superior
por ser el resultado del algoritmo de eliminación gaussiana. Por su parte,
la matriz $\mat{M}$ es triangular inferior, con unos en la diagonal, y
cada posición $i,k$ debajo de la diagonal contiene el valor $-m_{i,k}$. Se
trata de una matriz inversible; su inversa es similar, pero los $m_{i,k}$
no aparecen cambiados de signo. Como
\[ \mat{U} = \mat{M} \cdot \mat{A}, \]
si llamamos $\mat{L} = \mat{M}^{-1}$, tenemos que
\[ \mat{L} \cdot \mat{U} = \mat{A}, \]
con $\mat{L}$ triangular inferior con unos en la diagonal. Se trata, por lo
tanto, de una factorización LU para $\mat{A}$.

De lo anterior se puede concluir que si una matriz admite el algoritmo de
eliminación gaussiana sin pivoteo, entonces tiene factorización LU:
si se puede ejecutar el algoritmo,
la factorización se obtiene considerando como $\mat{U}$ a la matriz
triangulada y construyendo $\mat{L}$ a partir de los multiplicadores
calculados en el proceso. De allí que la complejidad de obtener la
factorización también sea de orden cúbico.

% TODO: ¡NO ME QUEDA CLARO QUE VALGA LA VUELTA!

No todas las matrices admiten factorización LU. A modo de ejemplo,
consideremos una matriz $\mat{A} \in \reals^{n \times n}$. Si tiene
factorización LU, entonces existen $a, b, c, d \in \nats$ tales que
\[ \mat{A} = \mat{L} \cdot \mat{U} = \begin{bmatrix}
        1 & 0 \\
        a & 1
    \end{bmatrix} \cdot \begin{bmatrix}
        b & c \\
        0 & d
    \end{bmatrix} = \begin{bmatrix}
        b  & c    \\
        ab & ac+d
    \end{bmatrix}. \]
La matriz $\begin{bmatrix} 0 & 0 \\ 1 & 0 \end{bmatrix}$ no cumple esta
propiedad, por lo que no puede tener factorización LU.

Sin embargo, teniendo en cuenta que toda matriz admite eliminación gaussiana
con pivoteo, puede considerarse la matriz inversible $\mat{P}$ que resulta de
multiplicar todas las matrices elementales que representan las operaciones
del pivoteo.
Así $\mat{P} \cdot \mat{A} = \mat{A'}$, donde $\mat{A'}$ admite eliminación
gaussiana sin pivoteo, y por lo tanto, tiene factorización LU.
Luego, podemos escribir
\[ \begin{aligned}
    \mat{A}
    &= \mat{P}^{-1} \cdot \mat{A'} \\
    &= \mat{P}^{-1} \cdot \mat{L} \cdot \mat{U}.
\end{aligned} \]
Esta escritura, que siempre existe, se conoce como factorización PLU
de $\mat{A}$.

Con respecto a la unicidad de la factorización LU, no siempre es posible
garantizarla (por ejemplo, la matriz nula tiene infinitas factorizaciones
LU distintas). Sin embargo, si $\mat{A}$ es inversible y tiene factorización
LU, entonces esta es única. % TODO: Demostrar esto.

Los siguientes dos resultados, que se enuncian sin demostración, hacen
referencia a la existencia de factorización LU para ciertos tipos particulares
de matrices:
\begin{itemize}
\item Las matrices estrictamente diagonal-dominantes por filas, es decir,
    aquellas que cumplen que $\vert a_{i,i} \vert > \sum_{j=1, j \neq i}
    \vert a_{i,j}\vert$ para todo $i \in \{1,\dots,n\}$, siempre admiten
    factorización LU.
\item Las matrices inversibles admiten factorización LU si y solo si todos
    sus menores principales son no nulos.
% TODO: Menores principales recién se define en la sección siguiente.
\end{itemize}
