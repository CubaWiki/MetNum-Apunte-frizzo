% -*- root: apunte-metodos.tex -*-

\section{Cálculo de autovalores}
\label{section:autovalores}

El problema de hallar los autovalores de una matriz es recurrente, y tiene
múltiples aplicaciones. Una de ellas ya se presentó anteriormente: computar
la descomposición en valores singulares de una matriz. También se los
utiliza en estadística para realizar análisis de componentes principales
sobre conjuntos de datos; permiten diagonalizar matrices, lo cual facilita
su exponenciación; y un variado etcétera que convierte su búsqueda
en un problema significativo para resolver.

\subsection{Método de la potencia}

Supongamos que tenemos una matriz $\mat{A} \in \reals^{n \times n}$
diagonalizable; es decir, $\mat{A}$ posee $n$ autovalores reales. Podemos
numerar estos autovalores de forma tal que
\[ \Vert \lambda_1 \Vert \geq \Vert \lambda_2 \Vert \geq \dots
    \geq \Vert \lambda_n \Vert. \]
Si la primera desigualdad es estricta, es decir, si $\mat{A}$ tiene un
autovalor simple $\lambda_1$ cuyo valor absoluto es mayor que el de todos los
demás, decimos que $\lambda_1$ es el \textbf{autovalor principal} de
$\mat{A}$. En ese caso, podemos aplicar el \textbf{método de la potencia} para
obtener su valor.

El método de la potencia es iterativo, y se basa en una idea sumamente
sencilla. Consideremos una base de autovectores de $\mat{A}$, $\{\vec{v}_1,
\cdots, \vec{v}_n\}$, ordenados de forma tal que cada $\vec{v}_i$ está
asociado al autovalor $\lambda_i$. Como punto de inicio de la iteración,
tomaremos \emph{a priori} un vector $\vec{x}^{(0)} \in \reals^n$ arbitrario.
Podemos escribir
\[ \vec{x}^{(0)} = \alpha_1 \cdot \vec{v}_1
    + \cdots + \alpha_n \cdot \vec{v}_n. \]

En cada iteración, simplemente multiplicaremos a izquierda por $\mat{A}$.
Es decir,
\[ \begin{aligned}
    \vec{x}^{(k)} = \mat{A} \cdot \vec{x}^{(k-1)}
        &= \mat{A}^k \cdot \vec{x}^{(0)} \\
        &= \mat{A}^k \cdot \left( \alpha_1 \cdot \vec{v}_1
            + \cdots + \alpha_n \cdot \vec{v}_n \right) \\
        &= \alpha_1 \cdot \mat{A}^k \cdot \vec{v}_1
            + \cdots + \alpha_n \cdot \mat{A}^k \cdot \vec{v}_n \\
        &= \alpha_1 \cdot \lambda_1^k \cdot \vec{v}_1
            + \cdots + \alpha_n \cdot \lambda_n^k \cdot \vec{v}_n \\
        &= \lambda_1^k \cdot \left( \alpha_1 \cdot \vec{v}_1
            + \alpha_2 \cdot
                \left(\frac{\lambda_2}{\lambda_1}\right)^k \cdot \vec{v}_2
            + \cdots + \alpha_n \cdot
                \left(\frac{\lambda_n}{\lambda_1}\right)^k \cdot \vec{v}_n
            \right).
\end{aligned} \]

Ahora bien, como para todo $i \in \{1,\dots,n\}$ se cumple que
$\lvert \lambda_1 \rvert > \lvert \lambda_i \rvert$, entonces
\[ \lim_{k\to\infty} \left(\frac{\lambda_i}{\lambda_1}\right)^k = 0. \]
Si llamamos $\vec{r}^{(k)} = \dfrac{\vec{x}^{(k)}}{\lambda_1^k}$,
tenemos que
\[ \lim_{k\to\infty} \vec{r}^{(k)} =
    \lim_{k\to\infty} \left( \alpha_1 \cdot \vec{v}_1
        + \alpha_2 \cdot
            \left(\frac{\lambda_2}{\lambda_1}\right)^k \cdot \vec{v}_2
        + \cdots + \alpha_n \cdot
            \left(\frac{\lambda_n}{\lambda_1}\right)^k \cdot \vec{v}_n
        \right) = \alpha_1 \cdot \vec{v}_1. \]

De aquí podemos extraer dos conclusiones, siempre y cuando $\alpha_1 \neq 0$:
\begin{itemize}
\item $\displaystyle \lim_{k\to\infty}
        \frac{\lVert \vec{x}^{(k)} \rVert}
             {\lVert \vec{x}^{(k-1)} \rVert}
    = \lim_{k\to\infty}
        \frac{\lvert \lambda_1^k\rvert}{\lvert \lambda_1^{k-1}\rvert} \cdot
        \frac{\lVert \vec{r}^{(k)} \rVert}{\lVert \vec{r}^{(k-1)} \rVert}
    = \lvert \lambda_1 \rvert$.

    Aquí $\lVert \bullet \lVert$ puede ser cualquier norma inducida; más aún,
    se puede considerar cualquier función que respete el producto por un
    escalar, tome siempre valores no negativos y se anule solo en el cero;
    una elección sencilla, por ejemplo, es la norma infinito.

    Por lo tanto, el cociente entre las normas de dos iteraciones sucesivas
    del método de las potencias converge al valor absoluto del autovalor
    principal de $\mat{A}$.
\item Para $k$ lo suficientemente grande, $\vec{x}^{(k)} \approx \lambda_1^k
    \cdot \alpha_1 \cdot \vec{v}_1$, que es múltiplo de $\vec{v}_1$ y, por
    lo tanto, es un autovector principal de $\mat{A}$.
\end{itemize}

En general, al aplicar el método de las potencias, puede resultar convienente
ir normalizando $\vec{x}^{(k)}$ en cada paso para evitar que sus componentes
alcancen valores extremos, lo cual provocaría que se pierda precisión.

La única hipótesis que el método requiere sobre $\vec{x}^{(0)}$ es que
$\alpha_1$, su componente en la dirección del autovector principal, no sea
nula. Esto suele ser difícil de garantizar, justamente porque no se conoce
dicho autovector. La solución suele ser elegir $\vec{x}^{(0)}$ de manera
arbitraria y modificar esta elección en caso de que el método no converja.

El método se puede completar con una técnica conocida como \textbf{deflación},
que permite, una vez conocido $\lambda_1$, seguir obteniendo autovalores de
$\mat{A}$ en orden de valor absoluto decreciente. Consiste en definir
\[\mat{A'} = \mat{A} - \lambda_1 \cdot \vec{u}_1 \cdot \vec{u}_1\trans, \]
donde $\vec{u}_1$ es un autovector unitario asociado a $\lambda_1$.
La matriz $\mat{A'}$ tiene autovalores $0, \lambda_2, \dots, \lambda_n$,
por lo que si $\lambda_2 > \lambda_3$, puede volver a aplicarse el método de
la potencia.

\subsection{Método de la potencia inversa}
El \textbf{método de la potencia inversa} es una variante del método de la
potencia que permite, dada una matriz inversible $\mat{A}$, encontrar su
autovalor (y autovector asociado) de módulo mínimo, si el mismo existe y
tiene multiplicidad simple.
Se basa en el hecho de que, si los autovalores de $\mat{A}$ son 
\[ \lambda_1, \dots, \lambda_n, \]
con $\lvert \lambda_1 \rvert < \lvert \lambda_i \rvert$ para todo $i \in
\{2,\dots,n\}$, entonces los autovalores de $\mat{A}^{-1}$ son
\[ \lambda_1^{-1}, \dots, \lambda_n^{-1}, \]
con $\lvert \lambda_1^{-1} \rvert > \lvert \lambda_i^{-1} \rvert$
para todo $i \in \{2,\dots,n\}$.
Por lo tanto, basta con aplicar el método de las potencias sobre
$\mat{A}^{-1}$ para obtener $\lvert \lambda_1^{-1} \rvert$.

Una variante interesante del método de la potencia inversa permite, dado un
valor $\mu \in \reals$, encontrar el autovalor de $\mat{A}$ más cercano a
$\mu$. Consiste en aplicar el método de la potencia sobre la matriz $(\mat{A}
- \mu \cdot \mat{I})^{-1}$, que tiene como autovalores a
\[ (\lambda_1 - \mu)^{-1}, \dots, (\lambda_n - \mu)^{-1}; \]
el autovalor $\lambda_i$ de $\mat{A}$ que minimiza la distancia con $\mu$ es
también el que maximiza el valor de $(\lambda_i - \mu)^{-1}$.

