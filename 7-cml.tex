% -*- root: apunte-metodos.tex -*-

\section{Cuadrados mínimos lineales}
\label{section:cml}

Un problema muy frecuente en distintos ámbitos consiste en, dados
un conjunto de puntos del plano $(x_1,y_1), \dots, (x_m,y_m)$ y una familia
de funciones $\mathcal{F}$ determinadas por ciertos parámetros, hallar
cuál de estas funciones cuyo gráfico se ajusta mejor (es decir, ``pasa más
cerca'') a dichos puntos. Esto permite, por ejemplo, considerar los resultados
de un experimento y predecir los valores que se obtendrían para distintos
valores de una variable involucrada.

Las familias de funciones que tendremos en cuenta serán siempre de la forma
\[ \mathcal{F} = \{ a_1 \cdot \varphi_1 + \dots + a_n \cdot \varphi_n
    : a_1, \dots, a_n \in \reals \} \]
donde $\varphi_1, \dots, \varphi_n$ son funciones reales fijas. A modo de
ejemplo, podemos considerar la familia de funciones lineales (en cuyo caso
tendremos dos parámetros), de funciones cuadráticas (donde habrá tres
parámetros), etc.

Existen varios criterios para cuantificar cómo se ajusta una función $f$ a un
determinado conjunto de puntos. Por ejemplo, tres de ellos son:
\begin{itemize}
\item Minimizar el \emph{máximo error} (\emph{minimax}) entre cada uno de los
    puntos y el gráfico de la función, es decir, considerar como métrica
    \[ \max_{1\leq i \leq m} \left\vert f(x_i) - x_i \right\vert. \]
    Este criterio tiene la desventaja ser muy susceptible a la presencia
    de \emph{outliers}.
\item Minimizar el \emph{error absoluto} entre los puntos y el gráfico de la
    función:
    \[ \sum_{i = 1}^{m} \left\vert f(x_i) - x_i \right\vert. \]
    Este método es mucho más estable ante \emph{outliers}, pero tiene la
    complicación de que se busca minimizar una función que no es
    diferenciable en el origen.
\item Minimizar el \emph{error cuadrático}:
    \[ \sum_{i = 1}^{m} \left( f(x_i) - x_i \right)^2, \]
    que es el criterio que vamos a utilizar. Se trata de un método que da
    un mayor peso relativo a los \emph{outliers}, pero sin permitir que
    dominen la métrica; además, tiene la ventaja de ser diferenciable en todo
    punto.
\end{itemize}

Entonces, queremos hallar valores para los parámetros $a_1, \dots, a_n$ que
realicen el mínimo
\[ \min_{a_1,\dots,a_n} \sum_{i=1}^{m} \big( a_1 \cdot \varphi_1(x_i)
    + \dots + a_n \cdot \varphi_n(x_i) - y_i \big)^2 \]
o, considerando,
\[ \mat{A} = \begin{bmatrix}
    \varphi_1(x_1) & \varphi_1(x_2) & \cdots & \varphi_1(x_m) \\
    \varphi_2(x_1) & \varphi_2(x_2) & \cdots & \varphi_2(x_m) \\
    \vdots         & \vdots         & \ddots & \vdots \\
    \varphi_n(x_1) & \varphi_n(x_2) & \cdots & \varphi_n(x_m) 
\end{bmatrix} \qquad
\vec{x} = \begin{bmatrix}
    a_1 \\ a_2 \\ \vdots \\ a_n
\end{bmatrix} \qquad
\vec{b} = \begin{bmatrix}
    y_1 \\ y_2 \\ \vdots \\ y_m
\end{bmatrix} \qquad \]
buscamos hallar $\vec{x} \in \reals^m$ que realice el mínimo
\[ \min_{\vec{x}} \left\Vert \mat{A} \cdot \vec{x} - \vec{b} \right\Vert_2^2. \]

A este último problema se lo conoce como el problema de \textbf{cuadrados
mínimos lineales}: hallar, dadas $\mat{A} \in \reals^{m \times n}$ y $\vec{b}
\in \reals^m$, un vector $\vec{x} \in \reals^n$ que minimice la distancia
entre $\mat{A} \cdot \vec{x}$ y $\vec{b}$.

Desde el punto de vista geométrico, existe una interpretación intuitiva para
este problema. Para todo $\vec{x} \in \reals^n$, el producto $\mat{A}
\cdot \vec{x}$ describe una combinación lineal de las columnas de $\mat{A}$;
en otras palabras, $\vec{x} \in \im(\mat{A})$, el subespacio generado por las
columnas de $\mat{A}$. De todos los elementos de este subespacio, buscamos
aquel que se encuentre a distancia mínima de $\vec{b}$. Este elemento siempre
existe y es único: se trata de la proyección ortogonal de $\vec{b}$ sobre el
subespacio $\im(\mat{A})$, es decir, $\proy_{\im(\mat{A})}(\vec{b})$.

Más formalmente, recordemos que siempre es posible escribir $\vec{b}$ como
$\vec{b} = \vec{b_1} + \vec{b_2}$, con
\begin{itemize}
\item $\vec{b_1} = \proy_{\im(\mat{A})}(\vec{b}) \in \im(\mat{A})$, y
\item $\vec{b_2} = \proy_{\im(\mat{A})^{\bot}}(\vec{b}) \in
    \im(\mat{A})^{\bot}$, el complemento ortogonal de la imagen de $\mat{A}$.
\end{itemize}
Entonces, es posible demostrar que $\vec{x}$ realiza el mínimo buscado,
y por lo tanto es solución del problema de cuadrados mínimos lineales,
si y solo si $\mat{A} \cdot \vec{x} = \vec{b_1}$.

Del razonamiento anterior pueden extraerse tres conclusiones:
\begin{enumerate}[label=(\roman*)]
\item Dado que $\vec{b_1} \in \im(\mat{A})$, seguro existe $\vec{x}$ tal que
    $\mat{A} \cdot \vec{x} = \vec{b_1}$. Por lo tanto, el problema de
    cuadrados mínimos lineales siempre tiene solución.
\item Como $\vec{b_1}$ es único, la solución es única si y solo si hay
    una única forma de escribir a $\vec{b_1}$ como combinación lineal de las
    columnas de $\mat{A}$. Esto equivale a pedir que las columnas de $\mat{A}$
    sean linealmente independientes, o que $\mat{A}$ sea de rango columna
    completo ($\rg(\mat{A}) = n$).
\item Si el sistema $\mat{A} \cdot \vec{x} = \vec{b}$ tiene alguna solución
    $\vec{x^\ast}$, entonces $\vec{b} \in \im(\mat{A})$, por lo que
    $\vec{b_1} = \vec{b}$, y por lo tanto
    $\vec{x^\ast}$ también será solución del problema de cuadrados mínimos
    lineales.
    Sin embargo, lo interesante de este problema es que tiene solución
    \emph{incluso} cuando el sistema $\mat{A} \cdot \vec{x} = \vec{b}$ no
    la tiene, es decir, cuando está \emph{sobredeterminado} (tiene ecuaciones
    que ``se contradicen''). En estos casos, obviamente, el resultado
    hallado no será una solución de $\mat{A} \cdot \vec{x} = \vec{b}$,
    pero sí será la mejor aproximación posible según el criterio de
    minimizar el error cuadrático.
\end{enumerate}

\subsection{Ecuaciones normales}
Una de las formas de encontrar la solución de un problema de cuadrados
mínimos es resolviendo el siguiente sistema de ecuaciones, que recibe el
nombre de \textbf{ecuaciones normales}:
\[ \mat{A}\trans \cdot \mat{A} \cdot \vec{x} = \mat{A}\trans \cdot \vec{b}. \]

Para verificar que estas ecuaciones resuelven el problema, es necesario tener
en cuenta en primer lugar que
$\nul(\mat{A}\trans) = \im(\mat{A})^{\bot}$; es decir, multiplicar a izquierda
por $\mat{A}\trans$ anula cualquier elemento que sea ortogonal a
$\im(\mat{A})$. Para comprender esto, basta notar que al tomar un
vector $\vec{v}$ y multiplicarlo a izquierda por $\mat{A}\trans$, se está
computando la suma de los productos internos entre $\vec{v}$ y cada una de
las filas de $\mat{A}\trans$, que son las columnas de $\mat{A}$; si $\vec{v}$
es ortogonal a $\im(\mat{A})$, entonces todos estos productos serán nulos.

Si, con este hecho en mente, observamos el lado izquierdo de la ecuación
anterior, y escribimos $\vec{b} = \vec{b_1} + \vec{b_2}$
como antes, tenemos que 
\[ \mat{A}\trans \cdot \vec{b}
    = \mat{A}\trans \cdot (\vec{b_1} + \vec{b_2})
    = \mat{A}\trans \cdot \vec{b_1} + \mat{A}\trans \cdot \vec{b_2}
    = \mat{A}\trans \cdot \vec{b_1}. \]
Por lo tanto, las ecuaciones normales equivalen a 
\[ \mat{A}\trans \cdot \mat{A} \cdot \vec{x} = \mat{A}\trans \cdot \vec{b_1}. \]

Esto quiere decir que cualquier solución de cuadrados mínimos, que debe
satisfacer $\mat{A} \cdot \vec{x} = \vec{b_1}$, también será solución de las
ecuaciones normales. Para ver que cualquier solución de las
ecuaciones normales es una solución de cuadrados mínimos, consideremos
$\vec{x}$ tal que $\mat{A}\trans \cdot \mat{A} \cdot \vec{x} = \mat{A}\trans \cdot \vec{b_1}$; en tal caso $\mat{A}\trans \cdot (\mat{A} \cdot \vec{x} -
\vec{b_1}) = 0$, por lo que
\[ \mat{A} \cdot \vec{x} - \vec{b_1} \in \nul(\mat{A}\trans) =
    \im(\mat{A})^{\bot}. \]
Por otra parte, como $\mat{A} \cdot \vec{x} \in \im(\mat{A})$ y
$\vec{b_1} \in \im(\mat{A})$, necesariamente
\[ \mat{A} \cdot \vec{x} - \vec{b_1} \in \im(\mat{A}). \]
El único vector que está simultaneamente en $\im(\mat{A})$ y en
$\im(\mat{A})^{\bot}$ es $0$. Luego $\mat{A} \cdot \vec{x} - \vec{b_1} = 0$,
por lo que $\mat{A} \cdot \vec{x} = \vec{b_1}$ y entonces $\vec{x}$ es una
solución de cuadrados mínimos.

A modo de curiosidad, las ecuaciones normales son exactamente las que se
obtienen si se calculan las derivadas parciales de la expresión del error
cuadrático en función de cada una de las componentes de $\vec{x}$ y se las
iguala a cero.

El sistema de ecuaciones que se obtiene al formular las ecuaciones normales
tiene la buena propiedad de que su matriz, $\mat{A}\trans \cdot \mat{A}$,
es semi-definida positiva, con lo cual puede usarse la factorización de
Cholesky para resolverlo de forma eficiente.

No obstante, las ecuaciones normales también presentan un inconveniente: no garantizan la estabilidad numérica.
El número de condición de la matriz $\mat{A}$, que ya de por sí puede no ser
bueno, se eleva al cuadrado al computar $\mat{A}\trans \cdot \mat{A}$.
Esto motiva la utilización de otros métodos más estables, que aprovechan
algunas de las factorizaciones matriciales estudiadas anteriormente.

\subsection{Resolución con factorización QR}
Un método para resolver el problema de cuadrados mínimos es utilizar la
factorización QR. Recordemos que toda matriz admite una factorización QR.
Si escribimos $\mat{A} = \mat{Q} \cdot \mat{R}$, la expresión que buscamos
minimizar se puede reescribir como
\[ \left\Vert \mat{A} \cdot \vec{x} - \vec{b} \right\Vert_2^2
    = \left\Vert \mat{Q} \cdot \mat{R} \cdot \vec{x} -
        \vec{b} \right\Vert_2^2. \]
Pero como $\mat{Q}$ es una matriz ortogonal,
multiplicar por $\mat{Q}\trans$ no altera la norma 2, por lo que resulta que
\[ \left\Vert \mat{A} \cdot \vec{x} - \vec{b} \right\Vert_2^2
    = \left\Vert \mat{Q}\trans \cdot \left( \mat{Q} \cdot \mat{R}
        \cdot \vec{x} - \vec{b} \right) \right\Vert_2^2.
    = \left\Vert \mat{R} \cdot \vec{x} - \mat{Q}\trans \cdot \vec{b}
        \right\Vert_2^2. \]

En definitiva, nuestro problema se convierte en hallar $\vec{x}$ que realice
el mínimo
\[ \min_{\vec{x}} \left\Vert \mat{R} \cdot \vec{x}
    - \mat{Q}\trans \cdot \vec{b} \right\Vert_2^2. \]

Para hallar este mínimo hay que proceder de una forma ligeramente distinta
en base a dos casos, que dependen de $k = \rg(\mat{A})$. En ambos casos
es necesario tener en cuenta que, como $\mat{Q}$ es inversible,
entonces $\rg(\mat{R}) = \rg(\mat{A}) = k$.

\begin{enumerate}[label=(\roman*)]
\item Si $\mat{A}$ es de rango columna completo, podemos escribir:
    \[ \mat{R} = \begin{sbmatrix}{.2cm}
        \matbox{.95cm}{1.5cm}{\mat{R_1}} \\ \hline
        \matbox{.55cm}{1.5cm}{\mat{0}}
    \end{sbmatrix} \hspace{2.5cm}
    \mat{Q}\trans \cdot \vec{b} = \begin{sbmatrix}{.2cm}
        \vvecbox{.95cm}{\vec{c}} \\ \hline
        \vvecbox{.55cm}{\vec{d}}
    \end{sbmatrix} \]
    donde $\mat{R_1} \in \reals^{n \times n}$ es triangular superior,
    $\vec{c} \in \reals^n$ y $\vec{d} \in \reals^{m-n}$.

    Entonces, resulta que:
    \[ \begin{aligned} \min_{\vec{x}} \left\Vert \mat{R} \cdot \vec{x}
            - \mat{Q}\trans \cdot \vec{b} \right\Vert_2^2
        &= \min_{\vec{x}} \left\Vert \begin{sbmatrix}{.2cm}
                \vvecbox{.95cm}{\mat{R_1} \cdot \vec{x}} \\ \hline
                \vvecbox{.55cm}{\mat{0}}
            \end{sbmatrix} - \begin{sbmatrix}{.2cm}
                \vvecbox{.95cm}{\vec{c}} \\ \hline
                \vvecbox{.55cm}{\vec{d}}
            \end{sbmatrix} \right\Vert_2^2
        = \min_{\vec{x}} \left\Vert \begin{sbmatrix}{.2cm}
                \vvecbox{.95cm}{\mat{R_1} \cdot \vec{x} - \vec{c}} \\ \hline
                \vvecbox{.55cm}{\vec{-d}}
            \end{sbmatrix} \right\Vert_2^2 \\[.4cm]
        &= \min_{\vec{x}} \left\Vert \mat{R_1} \cdot \vec{x}
            - \vec{c} \right\Vert_2^2 + \left\Vert \vec{d} \right\Vert_2^2
    \end{aligned} \]

    Basta con minimizar el primer término de la expresión, ya que es el único
    que depende de $\vec{x}$. En efecto, el mínimo se alcanza si dicho término
    se anula, es decir, si $\vec{x}$ es solución del sistema
    \[ \mat{R_1} \cdot \vec{x} = \vec{c}, \]
    que siempre tiene solución única por ser $\mat{R_1}$ cuadrada y de rango
    completo.

\item Si $\mat{A}$ no es de rango columna completo, es decir,
    $k < n$, necesitaremos que la
    factorización QR haya sido obtenida con \emph{pivoteo de columnas}; esto
    quiere decir, que haya sido construida de forma tal que las columnas
    incompletas de $\mat{R}$ se encuentren todas al final. En general, se permutan las columnas de $\mat{R}$ de forma tal que
    $\vert r_{1,1} \vert \geq \vert r_{2,2} \vert \geq \dots \geq
    \vert r_{n,n} \vert$. Así,
    \[ \mat{A} = \mat{Q} \cdot \mat{R} \cdot \mat{P}, \]
    donde $\mat{P}$ es la matriz de permutación correspondiente al pivoteo.
    Por lo tanto, el mínimo buscado será
    \[ \min_{\vec{x}} \left\Vert \mat{R} \cdot \mat{P} \cdot \vec{x}
            - \mat{Q}\trans \cdot \vec{b} \right\Vert_2^2
        = \min_{\vec{x}} \left\Vert \mat{R} \cdot \vec{\tilde x}
            - \mat{Q}\trans \cdot \vec{b} \right\Vert_2^2 \]
    donde $\vec{\tilde x} = \mat{P} \cdot \vec{x}$. Resolveremos el
    sistema para $\vec{\tilde x}$; terminado el proceso, deberemos tener
    en cuenta que las soluciones halladas tendrán permutadas sus componentes.

    Podemos escribir, entonces:
    \[ \mat{R} = \begin{sbmatrix}{.2cm}
        \matbox{.85cm}{1.2cm}{\mat{R_1}} & \matbox{.85cm}{.6cm}{\mat{R_2}} \\ \hline
        \matbox{.65cm}{1.2cm}{\mat{0}} & \matbox{.65cm}{.6cm}{\mat{0}} \\
    \end{sbmatrix} \hspace{1.5cm}
    \mat{Q}\trans\vec{b} = \begin{sbmatrix}{.2cm}
        \vvecbox{.85cm}{\vec{c}} \\ \hline
        \vvecbox{.65cm}{\vec{d}}
    \end{sbmatrix} \hspace{1.5cm}
    \vec{\tilde x} = \begin{sbmatrix}{.2cm}
        \vvecbox{.85cm}{\vec{x_1}} \\ \hline
        \vvecbox{.65cm}{\vec{x_2}}
    \end{sbmatrix} \]
    donde $\mat{R_1} \in \reals^{k \times k}$ es triangular superior,
    $\mat{R_2} \in \reals^{k \times n-k}$,
    $\vec{c}, \vec{x_1} \in \reals^{k}$ y
    $\vec{d}, \vec{x_2} \in \reals^{n-k}$.

    De lo anterior,
    \[ \begin{aligned} \min_{\vec{x}} \left\Vert \mat{R} \cdot \vec{x}
            - \mat{Q}\trans \cdot \vec{b} \right\Vert_2^2
        &= \min_{\vec{x}} \left\Vert \begin{sbmatrix}{.2cm}
                \vvecbox{.85cm}{\mat{R_1} \cdot \vec{x_1} + \mat{R_2} \cdot \vec{x_2} - \vec{c}} \\ \hline
                \vvecbox{.65cm}{\vec{-d}}
            \end{sbmatrix} \right\Vert_2^2 \\[.4cm]
        &= \min_{\vec{x}} \left\Vert \mat{R_1} \cdot \vec{x_1}
            + \mat{R_2} \cdot \vec{x_2}
            - \vec{c} \right\Vert_2^2 + \left\Vert \vec{d} \right\Vert_2^2
    \end{aligned} \]

    De nuevo, esta expresión alcanza el mínimo cuando el primer término es
    nulo. Las soluciones son infinitas: $\vec{x_2}$ puede fijarse libremente
    en cualquier $\vec{x_2^\ast}$, y luego se determina $\vec{x_1}$ como la
    única solución del sistema
    \[ \mat{R_1} \cdot \vec{x_1} = \vec{c} - \mat{R_2} \cdot \vec{x_2^\ast}. \]
    \end{enumerate}

Como ya se mencionó cuando se habló de la factorización QR como método
para resolver sistemas de ecuaciones lineales, se trata de un método
numéricamente muy estable, debido a que la matriz $\mat{Q}$ es ortogonal.
Sin embargo, al utilizarlo con matrices de rango incompleto, se vuelve
necesario incorporar el pivoteo; esto complejiza el algoritmo y
lo vuelve poco estable. Además, hallar el rango de $\mat{A}$ puede resultar
difícil debido a errores de redondeo.

\subsection{Resolución con descomposición en valores singulares}

Otra posibilidad para resolver el problema de cuadrados mínimos es utilizar
la descomposición en valores singulares de $\mat{A} = \mat{U} \cdot
\mat{\Sigma} \cdot \mat{V}\trans$.
En este caso, como $\mat{U}$ es ortogonal, multiplicar por
$\mat{U}\trans$ no modifica la norma 2, y tenemos que:
\[ \left\Vert \mat{A} \cdot \vec{x} - \vec{b} \right\Vert_2^2
    = \left\Vert \mat{U} \cdot \mat{\Sigma} \cdot \mat{V}\trans \cdot \vec{x}
        - \vec{b} \right\Vert_2^2
    = \left\Vert \mat{\Sigma} \cdot \mat{V}\trans \cdot \vec{x}
        - \mat{U}\trans \cdot \vec{b} \right\Vert_2^2. \]

Si llamamos $\mat{V}\trans \cdot \vec{x} = \vec{y}$, como $\mat{V}\trans$ es
inversible, tenemos un sistema de ecuaciones determinado: si encontramos un
valor que nos sirva para $\vec{y}$, podemos determinar cuál debe ser el valor
de $\vec{x}$. Entonces, sustituyendo, obtenemos que
\[ \left\Vert \mat{A} \cdot \vec{x} - \vec{b} \right\Vert_2^2
    = \left\Vert \mat{\Sigma} \cdot \vec{y}
        - \mat{U}\trans \cdot \vec{b} \right\Vert_2^2, \]
y, por lo tanto, el problema de minimización a resolver es
\[ \min_{\vec{x}} \left\Vert \mat{\Sigma} \cdot \vec{y}
        - \mat{U}\trans \cdot \vec{b} \right\Vert_2^2. \]

Volvemos a separar en dos casos según $k = \rg(\mat{A}) = \rg(\mat{\Sigma})$.

\begin{enumerate}[label=(\roman*)]
\item Si $\mat{A}$ es de rango columna completo, podemos escribir:
    \[ \mat{\Sigma} = \begin{sbmatrix}{.2cm}
        \sigma_1 &        &          \\
                 & \ddots &          \\
                 &        & \sigma_n \\ \hline
                 & \vvecbox{.55cm}{\mat{0}}
    \end{sbmatrix} \hspace{2.5cm}
    \mat{U}\trans \cdot \vec{b} = \begin{sbmatrix}{.2cm}
        \vvecbox{.95cm}{\vec{c}} \\ \hline
        \vvecbox{.55cm}{\vec{d}}
    \end{sbmatrix} \]
    donde $\vec{c} \in \reals^n$ y $\vec{d} \in \reals^{m-n}$.
    Así,
    \[ \begin{aligned}
        \min_{\vec{y}} \left\Vert \mat{\Sigma} \cdot \vec{y}
            - \mat{U}\trans \cdot \vec{b} \right\Vert_2^2
        &= \min_{\vec{y}} \left\Vert \begin{sbmatrix}{.2cm}
                \sigma_1 \cdot y_1 \\
                \vdots             \\
                \sigma_n \cdot y_n \\ \hline
                \vvecbox{.55cm}{\mat{0}}
            \end{sbmatrix} - \begin{sbmatrix}{.2cm}
                \vvecbox{.95cm}{\vec{c}} \\ \hline
                \vvecbox{.55cm}{\vec{d}}
            \end{sbmatrix} \right\Vert_2^2 \\
        &= \min_{\vec{y}} \left\Vert \begin{bmatrix}
                \sigma_1 \cdot y_1 - c_1 \\
                \vdots             \\
                \sigma_n \cdot y_n - c_n
            \end{bmatrix} \right\Vert_2^2
            + \left\Vert \vec{d} \right\Vert_2^2
    \end{aligned} \]

    Para alcanzar el mínimo basta con anular el primer término, lo
    cual sucede si y solo si se toma $\vec{y}
    = \left( \frac{c_1}{\sigma_1}, \dots, \frac{c_n}{\sigma_n} \right)$.

\item Si $\mat{A}$ no es de rango columna completo ($k < n$), solo las
    primeras $k$ entradas de la diagonal de $\mat{\Sigma}$ son no nulas.
    Escribimos entonces:
    \[ \mat{U}\trans \cdot \vec{b} = \begin{sbmatrix}{.2cm}
        \vvecbox{.85cm}{\vec{c}} \\ \hline
        \vvecbox{.65cm}{\vec{d}}
    \end{sbmatrix} \]
    con $\vec{c} \in \reals^k$ y $\vec{d} \in \reals^{m-k}$.
    Ahora,
    \[ \begin{aligned}
        \min_{\vec{y}} \left\Vert \mat{\Sigma} \cdot \vec{y}
            - \mat{U}\trans \cdot \vec{b} \right\Vert_2^2
        &= \min_{\vec{y}} \left\Vert \begin{sbmatrix}{.2cm}
                \sigma_1 \cdot y_1 \\
                \vdots             \\
                \sigma_k \cdot y_k \\ \hline
                \vvecbox{.55cm}{\mat{0}}
            \end{sbmatrix} - \begin{sbmatrix}{.2cm}
                \vvecbox{.95cm}{\vec{c}} \\ \hline
                \vvecbox{.55cm}{\vec{d}}
            \end{sbmatrix} \right\Vert_2^2 \\
        &= \min_{\vec{y}} \left\Vert \begin{bmatrix}
                \sigma_1 \cdot y_1 - c_1 \\
                \vdots             \\
                \sigma_k \cdot y_k - c_k
            \end{bmatrix} \right\Vert_2^2
            + \left\Vert \vec{d} \right\Vert_2^2
    \end{aligned} \]

    De nuevo, para alcanzar el mínimo, debe anularse el primer término.
    Existen infinitas soluciones, las cuales se logran tomando $\vec{y}
    = \left( \frac{c_1}{\sigma_1}, \dots, \frac{c_k}{\sigma_k},
        y_{k+1}, \dots, y_{n} \right)$, con
    $y_{k+1}, \dots, y_{n} \in \reals$ cualesquiera. Una posibilidad es
    tomarlos a todos iguales a 0, lo cual minimiza la norma de la solución
    $\vec{x}$ que se encontrará luego.
\end{enumerate}

En ambos casos, una vez hallado $\vec{y}$, resta resolver el sistema
$\mat{V}\trans \vec{x} = \vec{y}$ para hallar la solución $\vec{x}$
al problema de cuadrados mínimos.

Utilizar SVD para resolver el problema es más costoso que hacerlo mediante QR,
si bien existen optimizaciones posibles, como no computar la matriz $\mat{U}$
de forma explícita. Más allá de las consideraciones de eficiencia, en casos de
matrices de rango incompleto, su estabilidad numérica hace claramente
preferible emplear SVD por encima de QR.