% -*- root: apunte-metodos.tex -*-

\section{Cuadrados mínimos lineales}
\label{section:cml}

Un problema muy frecuente en distintos ámbitos consiste en, dados
un conjunto de puntos del plano $(x_1,y_1), \dots, (x_m,y_m)$ y una familia
de funciones $\mathcal{F}$ determinadas por ciertos parámetros, hallar
cuál de estas funciones cuyo gráfico se ajusta mejor (es decir, ``pasa más
cerca'') a dichos puntos. Esto permite, por ejemplo, considerar los resultados
de un experimento y predecir los valores que se obtendrían para distintos
valores de una variable involucrada.

Las familias de funciones que tendremos en cuenta serán siempre de la forma
\[ \mathcal{F} = \{ a_i \cdot \varphi_1 + \dots + a_n \cdot \varphi_n
    : a_1, \dots, a_n \in \reals \} \]
donde $\varphi_1, \dots, \varphi_n$ son funciones reales fijas. Por ejemplo,
podemos considerar la familia de funciones lineales (en cuyo caso tendremos
dos parámetros), de funciones cuadráticas (donde habrá tres parámetros), etc.

Existen varios criterios para cuantificar cómo se ajusta una función $f$ a un
determinado conjunto de puntos. Por ejemplo, tres de ellos son:
\begin{itemize}
\item Minimizar el \emph{máximo error} (\emph{minimax}) entre cada uno de los
    puntos y el gráfico de la función, es decir, considerar como métrica
    \[ \max_{1\leq i \leq n} \left\vert f(x_i) - x_i \right\vert. \]
    Este criterio tiene la desventaja ser muy susceptible a la presencia
    de \emph{outliers}.
\item Minimizar el \emph{error absoluto} entre los puntos y el gráfico de la
    función:
    \[ \sum_{i = 1}^{n} \left\vert f(x_i) - x_i \right\vert. \]
    Este método es mucho más estable ante \emph{outliers}, pero tiene la
    complicación de que se busca minimizar una función que no es
    diferenciable en el origen.
\item Minimizar el \emph{error cuadrático}:
    \[ \sum_{i = 1}^{n} \left( f(x_i) - x_i \right)^2, \]
    que es el criterio que vamos a utilizar. Se trata de un método que da
    un mayor peso relativo a los \emph{outliers}, pero sin permitir que
    dominen la métrica; además, tiene la ventaja de ser diferenciable en todo
    punto.
\end{itemize}

Entonces, queremos hallar valores para los parámetros $a_1, \dots, a_k$ que
realicen el mínimo
\[ \min  \sum_{i=1}^{n} \big( a_i \cdot \varphi_1(x_i)
    + \dots + a_i \cdot \varphi_k(x_i) - y_i \big)^2 \]
o, considerando,
\[ \mat{A} = \begin{bmatrix}
    \varphi_1(x_1) & \varphi_1(x_2) & \cdots & \varphi_1(x_n) \\
    \varphi_2(x_1) & \varphi_2(x_2) & \cdots & \varphi_2(x_n) \\
    \vdots         & \vdots         & \ddots & \vdots \\
    \varphi_k(x_1) & \varphi_k(x_2) & \cdots & \varphi_k(x_n) 
\end{bmatrix} \qquad
x = \begin{bmatrix}
    a_1 \\ a_2 \\ \vdots \\ a_k
\end{bmatrix} \qquad
b = \begin{bmatrix}
    y_1 \\ y_2 \\ \vdots \\ y_n
\end{bmatrix} \qquad \]
buscamos hallar $x \in \reals^k$ que realice el mínimo
\[ \min \left(\Vert \mat{A} \cdot x - b \Vert_2\right)^2. \]

A este último problema, el de hallar, dadas $\mat{A} \in \reals^{m \times n}$.

\subsection{Ecuaciones normales}

\subsection{Resolución con factorización QR}
\subsection{Resolución con factorización SVD}
