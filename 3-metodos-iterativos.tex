% -*- root: apunte-metodos.tex -*-

\section{Resolución iterativa de sistemas de ecuaciones lineales}
\subsection{Métodos iterativos}
Un \textbf{método iterativo} es un procedimiento para la resolución de un
problema que se basa en construir una sucesión de elementos $\lbrace x_k
\rbrace_{k \in \nats}$ tal que $\lim_{k\to\infty} x_k = x^\ast$, donde $x^\ast$
es una solución del problema que se quiere resolver.
Esto permite construir un algoritmo que se aproxime progresivamente a una
solución calculando, en cada paso, el $k+1$-ésimo término de la sucesión a
partir del $k$-ésimo. De esta forma, mediante sucesivas iteraciones, puede
obtenerse una aproximación cada vez mejor a una solución del problema.

En el caso de la resolución de un sistema de ecuaciones lineales $\mat{A}
\cdot x = b$, se busca una sucesión de vectores $\lbrace x_k \rbrace_{k \in
\nats}$ que converja a una solución del sistema, es decir, a un valor $x^\ast$
tal que $\mat{A} \cdot x^\ast = b$.

Es necesario contar con un criterio que permita decidir cuándo la aproximación
ya es lo suficientemente buena, para así detener la ejecución del algoritmo.
Esto se conoce como \textbf{criterio de parada}. Algunos de los criterios
más comúnmente utilizados son:
\begin{enumerate}[label=(\roman*)]
\item Establecer una cantidad fija de iteraciones.
\item Fijar un valor $\varepsilon > 0$ y parar cuando $||x_{k+1} - x_k||
    < \varepsilon$, es decir, cuando el ritmo de convergencia sea lo
    suficientemente lento.
\end{enumerate}

\subsection{Análisis de convergencia}
\subsection{Método de Jacobi}
\subsection{Método de Gauss-Seidel}
