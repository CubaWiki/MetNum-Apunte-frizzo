% -*- root: apunte-metodos.tex -*-

\section{Resolución iterativa de sistemas de ecuaciones lineales}
\label{section:metodos-iterativos}

\subsection{Métodos iterativos}
Un \textbf{método iterativo} es un procedimiento para la resolución de un
problema que se basa en construir una sucesión de elementos $\lbrace x_k
\rbrace_{k \in \nats}$ tal que $\lim_{k\to\infty} x_k = x^\ast$, donde $x^\ast$
es una solución del problema que se quiere resolver.
Esto permite construir un algoritmo que se aproxime progresivamente a una
solución calculando, en cada paso, el $k+1$-ésimo término de la sucesión a
partir del $k$-ésimo. De esta forma, mediante sucesivas iteraciones, puede
obtenerse una aproximación cada vez mejor a una solución del problema.

En el caso de la resolución de un sistema de ecuaciones lineales $\mat{A}
\cdot x = b$, se busca una sucesión de vectores $\lbrace x_k \rbrace_{k \in
\nats}$ que converja a una solución del sistema, es decir, a un valor $x^\ast$
tal que $\mat{A} \cdot x^\ast = b$.

\subsection{Análisis de convergencia}
% radio espectral

\subsection{Método de Jacobi}

\subsection{Método de Gauss-Seidel}
