% -*- root: apunte-metodos.tex -*-

\section{Factorización de matrices}
\subsection{Factorización de Cholesky}

Sea $\mat{A} \in \reals^{n \times n}$. Decimos que $\mat{A}$ es
\textbf{simétrica} si $\mat{A} = \mat{A}\trans$. Además, si $\mat{A}$ es
simétrica, decimos que es \textbf{definida positiva} si
$\fall{x \in \reals^{n}} x \neq 0 \Rightarrow x\trans \cdot \mat{A} \cdot x > 0$.

Equivalentemente, las matrices simétricas y definidas positivas (s.d.p.)
pueden caracterizarse como aquellas matrices cuyos \textbf{menores
principales} son todos positivos. Los menores principales de una matriz
$\mat{A}$ son los determinantes de las submatrices cuadradas de $\mat{A}$ que
contienen su esquina superior izquierda.

Dada una matriz $\mat{A}$, llamamos \textbf{factorización de Cholesky} a su
escritura en la forma
\[ \mat{A} = \mat{L} \cdot \mat{L}\trans \]
donde $\mat{L}$ es una matriz triangular inferior con elementos positivos
en la diagonal. Podemos demostrar que una matriz tiene factorización de
Cholesky si y solo si es simétrica y definida positiva.

\subsection{Factorización QR}

Dos vectores $x, y \in \reals^{n}$ se dicen \textbf{ortogonales} ($x \bot y$)
si su producto interno $\langle x,y \rangle = 0$. Un conjunto de vectores es
\textbf{ortogonal} si sus elementos son ortogonales dos a dos. Un conjunto de
vectores es \textbf{ortonormal} si es ortogonal y la norma de todos sus
elementos es $1$. Los elementos de un conjunto ortonormal siempre son
linealmente independientes.

Decimos que una matriz $\mat{Q} \in \reals^{n \times n}$ es \textbf{ortogonal}
si sus columnas forman un conjunto ortonormal. Se puede demostrar, además, que
una matriz es ortogonal si y solo si es inversible y su transpuesta es igual a
su inversa, es decir, $\mat{Q}\trans = \mat{Q}^{-1}$. El producto de matrices
ortogonales es también ortogonal, es decir, si $\mat{Q_1}$ y $\mat{Q_2}$ son
ortogonales, $\mat{Q_1} \cdot \mat{Q_2}$ es ortogonal.

Las matrices ortogonales poseen la importante propiedad de estar asociadas a las
transformaciones lineales que \emph{preservan norma 2}. En otras palabras, una
matriz $\mat{Q} \in \reals^{n \times n}$ es ortogonal si y solo si, para todo
$x \in \reals^n$, $\lVert \mat{Q} \cdot x \rVert_2 = \lVert x \rVert_2$.
De esto se desprende también que el número de condición de una matriz
ortogonal (con respecto a la norma 2) es $\kappa(\mat{Q}) = 1$. Todo esto nos
indica que las matrices ortogonales son muy estables numéricamente, lo cual
las vuelve muy atractivas a la hora de resolver problemas en forma
computacional.

Dada una matriz $\mat{A} \in \reals^{n \times n}$, existen $\mat{Q} \in
\reals^{n \times n}$ ortogonal y $\mat{R} \in \reals^{n \times n}$ triangular
superior tales que
\[ \mat{A} = \mat{Q} \cdot \mat{R} \]
Esta forma se conoce como \textbf{factorización QR} de $\mat{A}$.

Una de las aplicaciones clave de esta factorización es la resolución
de sistemas de ecuaciones lineales. Si se tiene el sistema $\mat{A} \cdot x =
b$, utilizando la factorización QR, se lo puede reescribir como
\[\mat{Q} \cdot \mat{R} \cdot x = b\]
Al ser $\mat{Q}$ ortogonal, se puede multiplicar
a ambos lados por $\mat{Q}\trans$ y se tiene
\[\mat{Q}\trans \mat{Q} \cdot \mat{R} \cdot x = \mat{Q}\trans b\]
o lo que es lo mismo,
\[\mat{R} \cdot x = \mat{Q}\trans \cdot b\]
Como $\mat{R}$ es triangular superior, este sistema se puede resolver de forma
eficiente con sustitución hacia atrás, con la importante ventaja de que
$\mat{Q}\trans$ es sumamente estable, evitando la introducción de errores
numéricos al calcular el producto con $b$.

Otra aplicación importante de la factorización QR es en la resolución del
problema de cuadrados mínimos lineales, que será estudiado más adelante.

\subsubsection{Método de Householder (Reflexiones)}
\subsubsection{Método de Givens (Rotaciones)}
\subsection{Descomposición en valores singulares}

Sea $\mat{A} \in \reals^{m \times n}$ una matriz cualquiera, con $\rg(A) = r$.
Entonces, existen
\begin{itemize}
\item una base ortonormal $\lbrace v_1, v_2, \dots, v_n \rbrace$ de $\reals^n$, y
\item una base ortonormal $\lbrace u_1, u_2, \dots, u_m \rbrace$ de $\reals^m$
\end{itemize}
tales que
\[ \mat{U}\trans \cdot \mat{A} \cdot \mat{V} = \mat{\Sigma} = \begin{bmatrix}
\sigma_1 & 0        & \cdots   &    &          &          &          \\
0        & \sigma_2 &          &          &          &          &          \\
\vdots   &          & \ddots   &          &          &          &          \\
         &          &          & \sigma_r &          &          &          \\
         &          &          &          & 0        &          &          \\
         &          &          &          &          & \ddots   &          \\
         &          &          &          &          &          & 0
\end{bmatrix} \]
donde
\begin{itemize}
\item $\mat{V} = \begin{bmatrix}
            &     &        &     \\
        v_1 & v_2 & \dots  & v_n \\
            &     &        &     \\
    \end{bmatrix} \in \reals^{n \times n}$ y
    $\mat{U} = \begin{bmatrix}
            &     &        &     \\
        u_1 & u_2 & \dots  & u_m \\
            &     &        &     \\
    \end{bmatrix} \in \reals^{m \times m}$ son ortogonales, y
\item $\mat{\Sigma} \in \reals^{m \times n}$ es una matriz diagonal,
    con $\sigma_i > 0 \fall{i = 1, \dots, r}$.
\end{itemize}

Los valores $\sigma_i$ se denominan los \textbf{valores singulares} de \mat{A}.

\subsubsection{Demostración de la existencia}

Queremos hallar $\mat{U}$, $\mat{V}$, $\mat{\Sigma}$ que cumplan la igualdad
enunciada:
\[ \mat{U}\trans \cdot \mat{A} \cdot \mat{V} = \mat{\Sigma} \]
o, equivalentemente (usando la ortogonalidad de $\mat{U}$ y $\mat{V}$),
\begin{multicols}{2}\noindent
\[ \begin{aligned} \mat{A} \cdot \mat{V} &= \mat{U} \cdot \mat{\Sigma} \\
    & \Updownarrow \\
    \mat{A} \cdot \begin{bmatrix}
            &     &        &     \\
        v_1 & v_2 & \dots  & v_n \\
            &     &        &     \\
    \end{bmatrix} &= \begin{bmatrix}
            &     &        &     \\
        u_1 & u_2 & \dots  & u_m \\
            &     &        &     \\
    \end{bmatrix} \cdot \mat{\Sigma} \\
    & \Updownarrow
\end{aligned}
\]
\begin{numcases}{}
    \mat{A} \cdot v_i = \sigma_i \cdot u_i & para $i = 1,\dots,r$
        \label{eq:svd:caso1} \\
    \mat{A} \cdot v_i = 0                  & para $i = r+1,\dots,n$
        \label{eq:svd:caso2}
\end{numcases}

\columnbreak \noindent
\[\begin{aligned} \mat{A}\trans \cdot \mat{U} &= \mat{V} \cdot \mat{\Sigma} \\
    & \Updownarrow \\
    \mat{A}\trans \cdot \begin{bmatrix}
            &     &        &     \\
        u_1 & u_2 & \dots  & u_m \\
            &     &        &     \\
    \end{bmatrix} &= \begin{bmatrix}
            &     &        &     \\
        v_1 & v_2 & \dots  & v_n \\
            &     &        &     \\
    \end{bmatrix} \cdot \mat{\Sigma} \\
    & \Updownarrow
\end{aligned}
\]
\begin{numcases}{}
    \mat{A}\trans \cdot u_i = \sigma_i \cdot v_i & para $i = 1,\dots,r$
        \label{eq:svd:caso3} \\
    \mat{A}\trans \cdot u_i = 0                  & para $i = r+1,\dots,m$
        \label{eq:svd:caso4}
\end{numcases}

\end{multicols}

Buscaremos definir
\begin{itemize}
\item $v_1, \dots, v_n \in \reals^n$ ortonormales,
\item $u_1, \dots, u_m \in \reals^m$ ortonormales, y
\item $\sigma_1, \dots, \sigma_r > 0$
\end{itemize}
que cumplan las ecuaciones (\ref{eq:svd:caso1}), (\ref{eq:svd:caso2}),
(\ref{eq:svd:caso3}) y (\ref{eq:svd:caso4}).

En primer lugar, vemos que:
\begin{enumerate}[label=(\roman*)]
\item Para $i = 1, \dots, r$,
\[ \begin{aligned}
    \mat{A}\trans \cdot \mat{A} \cdot v_i &= \sigma_i \cdot \mat{A}\trans \cdot u_i \\
    \mat{A}\trans \cdot \mat{A} \cdot v_i &= \sigma_i^2 \cdot v_i \\
    \end{aligned} \]
es decir, los $v_i$ deberán ser autovectores de $\mat{A}\trans \cdot \mat{A}$,
con autovalor $\sigma_i^2$.
\item Para $i = r+1, \dots, n$,
\[ \mat{A}\trans \cdot \mat{A} \cdot v_i = \mat{A}\trans \cdot 0 = 0 \]
es decir, los $v_i$ deberán ser autovectores de $\mat{A}\trans \cdot \mat{A}$,
con autovalor $0$.
\end{enumerate}

Como $\mat{A}\trans \cdot \mat{A}$ es simétrica, existe una base ortonormal de
$\reals^n$ compuesta por autovectores de $\mat{A}\trans \cdot \mat{A}$.
Además, por ser semidefinida positiva, todos sus autovalores son no negativos.

Por otra parte, tenemos que
\[ \fall{x \in \reals^n} \mat{A} \cdot x = 0 \Leftrightarrow
    \mat{A}\trans \cdot \mat{A} \cdot x \]
dado que, trivialmente, $\mat{A} \cdot x = 0 \Rightarrow
\mat{A}\trans \cdot \mat{A} \cdot x = 0$, mientras que
$ \mat{A}\trans \cdot \mat{A} \cdot x = 0
    \Rightarrow x\trans \cdot \mat{A}\trans \cdot \mat{A} \cdot x = 0
    \Rightarrow \lVert \mat{A} \cdot x \rVert = 0
    \Rightarrow \mat{A} \cdot x = 0$.
Así, $\rg(\mat{A}\trans \cdot \mat{A}) = \rg(\mat{A}) = 0$, lo cual significa
que $\mat{A}\trans \cdot \mat{A}$ tiene exactamente $n - r$ autovectores
asociados al autovalor $0$.

Definimos entonces $\lbrace v_1, \dots, v_n \rbrace$ como una base ortonormal
de autovectores de $\mat{A}\trans \cdot \mat{A}$, ordenados de modo que los
últimos $n-r$ son los asociados al autovalor $0$. Veamos que estos $v_i$
cumplen con las condiciones necesarias.

\begin{enumerate}[label=(\roman*)]
\item Para $i = 1, \dots, r$, sea $\lambda_i$ el autovalor de $\mat{A}\trans
    \cdot \mat{A}$ asociado al autovector $v_i$. Sabemos que $\lambda_i > 0$.
    Definimos entonces:
    \begin{itemize}
    \item $\sigma_i = \sqrt{\lambda_i} > 0$.
    \item $\displaystyle u_i = \frac{\mat{A} \cdot v_i}{\sigma_i}$.
        Trivialmente, puede verse que satisfacen la ecuación
        (\ref{eq:svd:caso1}).
    \end{itemize}

    Veamos que $u_1, \dots, u_r$ son ortonormales:
    \begin{itemize}
    \item $\displaystyle \lVert u_i \rVert_2^2
        = \left( \frac{\mat{A} \cdot v_i}{\sigma_i} \right)\trans \cdot
          \left( \frac{\mat{A} \cdot v_i}{\sigma_i} \right)
        = \frac{1}{\sigma_i^2} (v_i\trans \cdot \mat{A}\trans
          \cdot \mat{A} \cdot v_i)
        = \frac{1}{\sigma_i^2} (v_i\trans \cdot \sigma_i^2 \cdot v_i)
        = v_i\trans \cdot v_i
        = \lVert v_i \rVert_2^2
        = 1$.
    \item $\displaystyle i \neq j \Rightarrow u_i\trans \cdot u_j
        = \left( \frac{\mat{A} \cdot v_i}{\sigma_i} \right)\trans \cdot
          \left( \frac{\mat{A} \cdot v_j}{\sigma_j} \right)
        = \frac{v_i\trans \cdot \mat{A}\trans \cdot \mat{A} \cdot v_j}
          {\sigma_i \cdot \sigma_j}
        = \frac{v_i\trans \cdot \sigma_j^2 \cdot v_j}
          {\sigma_i \cdot \sigma_j}
        = \frac{\sigma_j}{\sigma_i} \cdot v_i\trans \cdot v_j
        = 0$.
    \end{itemize}

    Verifiquemos, también, que cumplan la ecuación (\ref{eq:svd:caso3}):
    \[ \mat{A}\trans \cdot u_i
        = \frac{\mat{A}\trans \cdot \mat{A} \cdot v_i}{\sigma_i}
        = \frac{\sigma_i^2 \cdot v_i}{\sigma_i}
        = \sigma_i \cdot v_i \]

\item Para $i = r+1, \dots, n$, ya vimos que, como $\mat{A}\trans \cdot
    \mat{A} \cdot v_i = 0$, también $\mat{A} \cdot v_i = 0$, por lo que
    se satisface la ecuación (\ref{eq:svd:caso2}).

    Aún no definimos $u_{r+1}, \dots, u_m$. Lo hacemos de modo que sean una
    base ortonormal de $\nul(\mat{A}\trans)$. Así, satisfacen la ecuación
    (\ref{eq:svd:caso4}): $\mat{A}\trans \cdot u_i = 0$.

    Solo falta ver que $\lbrace u_1, \dots, u_m \rbrace$ es una base
    ortonormal de $\reals^m$. Ahora bien:
    \begin{itemize}
    \item es una base, dado que $\lbrace u_1, \dots, u_r \rbrace$ es una base de
        $\im(\mat{A})$, $\lbrace u_{r+1},\dots,u_m \rbrace$ es una base de
        $\nul(\mat{A}\trans)$, y $\im(\mat{A}) \oplus \nul(\mat{A}\trans) =
        \reals^m$.
    \item ya vimos que tanto $\lbrace u_1, \dots, u_r \rbrace$ como
        $\im(\mat{A})$, $\lbrace u_{r+1},\dots,u_m \rbrace$ son conjuntos
        ortonormales. Basta ver, entonces, que tomando $i = 1,\dots,r$ y
        $j = r+1,\dots,m$, se cumple $u_i\trans \cdot u_j = 0$. En efecto,
        $\displaystyle u_i\trans \cdot u_j
            = \left( \frac{\mat{A} \cdot v_i}{\sigma_i} \right)\trans \cdot u_j
            = \frac{1}{\sigma_i} \cdot v_i\trans \cdot \mat{A}\trans \cdot u_j
            = \frac{1}{\sigma_i} \cdot v_i\trans \cdot 0
            = 0$.

    \end{itemize}

\end{enumerate}




\subsubsection{Forma de Hessenberg}

